\documentclass[a4paper, 14pt, oneside]{extbook}
\usepackage{comment}
\usepackage{amsmath}
\usepackage{lipsum}
\usepackage[nottoc,notlot,notlof]{tocbibind}
\usepackage{natbib}
\usepackage{algorithmicx}
\usepackage{float}
\usepackage{rotating}
\usepackage{afterpage}
\usepackage[T1]{fontenc}
%\usepackage[utf8x]{inputenc}
\usepackage[italian]{babel}
\usepackage{geometry}
\usepackage{courier}
\usepackage[bookmarks, hidelinks]{hyperref}
\newgeometry{
left=   1.5 in,
bottom= 1.5 in,
right=  1 in,
top=    1 in
}

\usepackage{fancyhdr}
\usepackage[chapter, newfloat]{minted} % Code listings, with syntax highlighting
\usepackage[font=small]{caption}

\newenvironment{code}{\captionsetup{type=listing}}{}
\SetupFloatingEnvironment{listing}{name=Code}

%   Code list in content index
\renewcommand{\listoflistings}{
  \cleardoublepage
  \addcontentsline{toc}{chapter}{List of Code}
  \listof{listing}{List of Code}
}

\setminted{
    frame=lines,
    framesep=2mm,
    bgcolor=Snow1,
    fontsize=\small,
    linenos,
    breaklines
}

% Grafica
\usepackage{graphicx,pstricks}
\usepackage{graphics}
\graphicspath{{img/}}
\usepackage{xcolor} % to access the named colour LightGray
\definecolor{LightGray}{gray}{0.9}

% Package usati per il frontespizio
\usepackage{tikz}
\usepackage{pgf-pie}
\usepackage{pgfplots}
\pgfplotsset{width=7cm,compat=1.8}
\usetikzlibrary{patterns}
\usepackage{minted}

%Algorithm
\usepackage[noend]{algpseudocode}

\setlength\headheight{44.2pt}
%Page Style
\usepackage{setspace}
%\setstretch{2.5} 
\doublespace

%\cfoot{\thepage}
\lhead[]{}
\rhead[]{\leftmark}

\pagestyle{fancy}{
\lhead{\includegraphics[scale=0.3]{img/logo/hlogo.png}}
\rhead{\footnotesize{Titolo abbreviato come intestazione}}
}

%Other

%Testo riempitivo

%Titolo indice
\renewcommand*\contentsname{Indice}

%Inclusione bibliografia nell'indice
\renewcommand\bibname{Bibliografia}

\begin{document}

%\maketitle
\begin{titlepage}
\thispagestyle{empty}
\raggedright % Allinea a sinistra

\begin{tikzpicture}
\node[anchor=south west] at (4,0) {\includegraphics[scale=0.75]{img/logo/logo_copertina_1}};
\node[anchor=south west] at (0,1.5) {\includegraphics{img/logo/logo_copertina_2}};
\node[anchor=south west] at (0,0.5) {\textsf{Scuola Politecnica e delle Scienze di Base}};
\node[anchor=south west] at (0,0) {\textsf{Corso di Laurea in Ingegneria Informatica}};
\end{tikzpicture}

\vfill

{\large Elaborato in \textbf{Architettura dei Sistemi Digitali}}
\\[1cm]
{\textbf{\textit{\LARGE Elaborato finale}}}
\\[1cm]
{\large Anno Accademico 2024/25}

\vfill


\begin{table}[h]
{\raggedright Studenti}
\\
\textbf{Filomena Vigliotti}
\\
\textbf{matr. M63001734}
\\
\textbf{Ciro Scognamilgio}
\\
\textbf{matr. M6300}
\\
\textbf{Antonio Sirignano}
\\
\textbf{matr. M63001732}
\end{table}

\end{titlepage}
\frontmatter

%\input{dedica.tex}

{\setstretch{1.5}
\tableofcontents
}

\mainmatter

%\chapter*{Introduzione}

\addcontentsline{toc}{chapter}{Introduzione}

\lipsum[1-3]

% Esempio di citazioni bibliografiche - vedi file "bibliography.bib"
\cite{librocitato}
\cite{articolocitato}
\cite{articoloconferenza}
\cite{sitocitato}
\chapter{Esercizio 1}

\section{Multiplexer 16:1}
Un multiplexer è una \textbf{macchina combinatoria}, ovvero una macchina la cui uscita in un determinato istante di tempo dipende solo dall'ingresso nel medesimo istante, e quindi realizza una funzione del tipo:
\begin{equation*}
    U = f(I)
\end{equation*}
dove $I$ e $U$ rappresentano rispettivamente gli insiemi limitati dei valori di ingresso e di uscita.\\
Il Multiplexer realizza una connessione \textit{n:1}, ovvero connette $n$ sorgenti a un'unica destinazione sulla base di segnali di selezione.\\
Un \textbf{Multiplexer lineare} è composto da $n$ segnali in ingresso e $n$ segnali di selezione. Tale dispositivo convoglia uno specifico segnale in ingresso verso l'uscita solo se il corrispondente segnale di selezione è alto. Uno svantaggio di un dispositivo di questo tipo è il numero eccessivo di fili per i segnali di selezione. Per risolvere ciò si può aggiungere un \textbf{Decoder}, un altro dispositivo notevole, che riceve in ingresso una parola codice di $n$ bit e presenta in uscita la sua rappresentazione decodificata di \(2^n\) bit.\\
Unendo un Multiplexer lineare a un Decoder, l'architettura diventa quella in figura, e si ottiene un componente definito \textbf{Multiplexer indirizzabile}, che diversamente da quello lineare, prende solo 2 segnali di selezione in ingresso. Un MUX indirizzabile è a sua volta una macchina notevole, caratterizzata da \(2^n\) ingressi, $n$ ingressi di selezione e un'unica uscita.
\begin{figure}[H]
	\centering
	\includegraphics[width=0.4\textwidth]{img/mux_indirizz}
	\caption{multiplexer indirizzabile}
	\label{mux_16:1} 
\end{figure}
Si vuole ora progettare un multiplexer indirizzabile 16:1, utilizzando un approccio per composizione, a partire da multiplexer 4:1.\\
Tale multiplexer è rappresentato di seguito.
\begin{figure}[H]
	\centering
	\includegraphics[width=0.3\textwidth]{img/mux_16-1}
	\caption{multiplexer 16:1}
	\label{mux_16:1} 
\end{figure}

\subsection{Progetto e architettura}
Dapprima si utilizza un approccio per composizione per realizzare un multiplexer 4:1 con multiplexer 2:1.\\
\begin{figure}[H]
	\centering
	\includegraphics[width=0.3\textwidth]{img/mux_2-1}
	\caption{multiplexer 2:1}
	\label{mux_2:1} 
\end{figure}
Il primo componente che si realizza è un multiplexer 2:1, caratterizzato dalla seguente tabella di verità:
\begin{table}[h!]
    \centering
    \begin{tabular}{||c|c|c||c||}
        \hline
        \hline
         $\mathbf{s_0}$ & $\mathbf{i_1}$ & $\mathbf{i_0}$ & $\mathbf{u}$\\
         \hline
         0 & 0 & 0 & 1 \\
         \hline
         0 & 0 & 1 & 1 \\
         \hline
         0 & 1 & 0 & 0 \\
         \hline
         0 & 1 & 1 & 1 \\
         \hline
         1 & 0 & 0 & 0 \\
         \hline
         1 & 0 & 1 & 0 \\
         \hline
         1 & 1 & 0 & 1 \\
         \hline
         1 & 1 & 1 & 1 \\
         \hline
         \hline 
    \end{tabular}
    \caption{Tabella di verità di un Mux 2:1}
    \label{tab:mux_2:1}
\end{table}
    
%\begin{figure}[H]
%	\centering
%	\includegraphics[width=0.3\textwidth]{img/tabVeritaMux_2_1}
%	\caption{Tabella di verità di un Mux 2:1}
%	\label{tab_mux_2:1} 
%\end{figure}
da cui si ottiene l'equazione:
\begin{center}
    \begin{equation*}
          u = (i_0 \text{ AND } \bar{s_0}) \text{ OR } (i_1 \text{ AND } s_0)
    \end{equation*}
\end{center}
Il successivo componente da costruire è un multiplexer 4:1.
\begin{figure}[H]
	\centering
	\includegraphics[width=0.3\textwidth]{img/mux_4-1}
	\caption{multiplexer 4:1}
	\label{mux_4:1} 
\end{figure}
Per composizione, a partire da 3 multiplexer 2:1, si può ottenere un multiplexer 4:1
\begin{figure}[H]
	\centering
	\includegraphics[width=0.7\textwidth]{img/mux_4-1_comp}
	\caption{multiplexer 4:1 per composizione di multiplexer 2:1}
	\label{mux_4:1_comp} 
\end{figure}
I 4 ingressi entrano in due multiplexer 2:1, che prendono due ingressi e producono un'uscita ciascuno; tali uscite vengono immesse nel terzo multiplexer, che produrrà l'unico output finale. Concettualmente, si divide la selezione in ingresso al multiplexer esterno in due parti:
\begin{itemize}
    \item la parte meno significativa (indicata dal colore verde) \(s_0\), viene posta in ingresso ai multiplexer del primo stadio e seleziona per ciascuno un filo in uscita;
    \item la parte più significativa (indicata dal colore viola) \(s_1\) entra nel multiplexer del secondo stadio e decide quale dei due fili, provenienti dai due blocchi precedenti, sarà immessa in uscita.
\end{itemize}
In maniera analoga si procede con la progettazione del multiplexer 16:1.\\
Anche in questo caso, sono stati usati dei colori per identificare i collegamenti tra le componenti.
\begin{figure}[H]
	\centering
	\includegraphics[width=0.7\textwidth]{img/mux_16-1_composizione}
	\caption{multiplexer 16:1 per composizione di multiplexer 4:1}
	\label{mux_16:1_comp} 
\end{figure}


\subsection{Implementazione}
Per l'implementazione si procede con un approccio di tipo strutturale, iniziando quindi dalla codifica del multiplexer 2:1, e, a partire da questo si compongnono dispositivi sempre più complessi fino ad arrivare all'obiettivo del multiplexer 16:1.
\paragraph{Mux 2:1} Di seguito il codice riguardante il Mux 2:1.
\begin{code}
    \inputminted[frame=lines, framesep=2mm, baselinestretch=1.2, bgcolor=LightGray, fontsize=\footnotesize, linenos]{vhdl}{vhdl_files/mux_2_1.vhdl}
    \caption{Multiplexer 2:1 in VHDL}
    \label{lst:mux_2_1}
\end{code}
L'interfaccia del componente ha come ingressi \texttt{i0} ed \texttt{i1}, come selezione \texttt{s0} e come uscita \texttt{u}.\\
Al seguito della definizione dell'interfaccia, si definisce il comportamento dell'entità, che risponde alla tabella della verità \ref{tab:mux_2:1}.\\
\paragraph{Mux 4:1} Si prosegue con il Mux 4:1. \\
Come anticipato, viene costruito a partire da tre mux 2:1.
\begin{code}
    \inputminted[frame=lines, framesep=2mm, baselinestretch=1.2, bgcolor=LightGray, fontsize=\footnotesize, linenos]{vhdl}{vhdl_files/mux_4_1.vhdl}
    \caption{Multiplexer 4:1 in VHDL}
    \label{lst:mux_4_1}
\end{code}
%spiegazione codice 4:1
In quest'entità, l'interfaccia è dichiarata come segue:
\begin{itemize}
    \item Il parametro \texttt{i} vettore di 4 elementi, ognuno corrispondente ad un ingresso del mux 4:1.
    \item Il parametro \texttt{s} vettore di 2 elementi, ognuno corrispondente ad un ingresso di selezione.
    \item Il parametro \texttt{u} corrispondente all'uscita del multiplexer.
\end{itemize}
A seguire si definisce la struttura del mux 4:1, utilizzando mux 2:1 come componenti. \\
Con il ciclo \texttt{for}, vengono stanziati i primi due mux 2:1, i quali riceveranno in ingresso rispettivamente, gli ingressi del mux 4:1 e la loro uscita è il vettore d'appoggio \texttt{u\_mid}, il quale è talvolta l'ingresso del terzo mux 2:1.
\paragraph{Mux 16:1} In maniera analoga si procede con la costruzione del mux 16:1.
Il codice è il seguente:
\begin{code}
    \inputminted[frame=lines, framesep=2mm, baselinestretch=1.2, bgcolor=LightGray, fontsize=\footnotesize, linenos]{vhdl}{vhdl_files/mux_16_1.vhdl}
    \caption{Multiplexer 16:1 in VHDL}
    \label{lst:mux_16_1}
\end{code}

\subsection{Simulazione}
Per la simulazione, vi è la necessità di un testbench, il quale generiamo in maniera automatica tramite software appositi.\\
In tale progetto la generazione viene effettuata tramite ChatGPT ed il codice è il seguente:
\begin{code}
    \inputminted[frame=lines, framesep=2mm, baselinestretch=1.2, bgcolor=LightGray, fontsize=\footnotesize, linenos]{vhdl}{vhdl_files/tb_mux_16_1.vhdl}
    \caption{Testbench multiplexer 16:1 in VHDL}
    \label{lst:tb_mux_16_1}
\end{code}
 Una volta generato ciò, utilizzando i software GHDL e GTKWAVE, vengono eseguiti i seguenti comandi:
 \begin{figure}[H]
	\centering
	\includegraphics[width=1\textwidth]{img/commands_sim_16_1}
	\caption{Comandi per la simulazione}
	\label{comandi_sim_mux_16:1} 
\end{figure}
Con l'esecuzione dell'ultimo comando, vi si apre una nuova finestra che permette la visualizzazione delle onde:
 \begin{figure}[H]
	\centering
	\includegraphics[width=1\textwidth]{img/waves_sim_16_1}
	\caption{Risultati della simulazione: waveform}
	\label{comandi_sim_mux_16:1} 
\end{figure}

\subsection{Implementazione 2.0}
In alcuni casi, può essere utile specificare i singoli ingressi, in particolare quando gli ingressi provengono da fonti diverse; in tal caso, è preferibile l'implementazione che segue:
\begin{code}
    \inputminted[frame=lines, framesep=2mm, baselinestretch=1.2, bgcolor=LightGray, fontsize=\footnotesize, linenos]{vhdl}{vhdl_files/mux16_1_singIng.vhd}
    \caption{Multiplexer 16:1 in VHDL: ingressi trattati separatamente}
    \label{lst:mux_16_1}
\end{code}
Ovviamente, la macchina sarà fatta allo stesso modo, come si può vedere dallo schematic generato da Vivado:
 \begin{figure}[H]
	\centering
	\includegraphics[width=0.8\textwidth]{img/mux16_1_vivado}
	\caption{Schematic Vivado: Mux 16:1}
	\label{Schematic Vivado: Mux 16:1} 
\end{figure}
Il multiplexer lavorerà allo stesso modo, con gli stessi risultati simulativi.


\section{Rete di interconnessione a 16 ingressi e 4 uscite}
Una rete di interconnessione è è un tipo di rete di commutazione che permette di instradare i segnali da un insieme di ingressi a un insieme più ridotto di uscite. Tale rete può essere progettata attraverso un adeguato utilizzo di Multiplexer e Demultiplexer. \\
Nel caso in esame, si vuole progettare una rete che prenda 16 ingressi e restituisca 4 uscite. Si utilizza anche in questo caso un approccio per composizione, a partire dal Multiplexer 16:1 implementato nell'esercizio precedente, la cui uscita sarà posta in ingresso a un Demultiplexer 1:4.\\
La rete complessiva sarà fatta in questo modo:
 \begin{figure}[H]
	\centering
	\includegraphics[width=1\textwidth]{img/Rete_interconn}
	\caption{Rete di interconnesione}
	\label{Demux 1:2} 
\end{figure}


\subsection{Progettazione}
 Anche in questo caso, prima di procedere all'implementazione della rete nel complesso, si costruisce il Demultiplexer 4:1 a partire da Demultiplexer 2:1.\\
Un Demultiplexer $1:u$ è un dispositivo che prende un solo segnale di ingresso, due segnali di selezione e a partire da essi restituisce $u$ uscite. \\
Un Demultiplexer 2:1 è un dispositivo fatto in questo modo:
 \begin{figure}[H]
	\centering
	\includegraphics[width=0.5\textwidth]{img/demux_1_2}
	\caption{Demux 1:2}
	\label{Demux 1:2} 
\end{figure}
Tale componente è caratterizzato dalla seguente tabella di verità:
\begin{table}[H]
    \centering
    \begin{tabular}{||c|c|c||c||}
        \hline
        \hline
         $\mathbf{s_0}$ & $\mathbf{i_0}$ & $\mathbf{u_0}$ & $\mathbf{u_1}$\\
         \hline
         0 & 0 & 0 & 0 \\
         \hline
         0 & 1 & 1 & 0 \\
         \hline
         1 & 0 & 0 & 0 \\
         \hline
         1 & 1 & 0 & 1 \\
         \hline
         \hline 
    \end{tabular}
    \caption{Tabella di verità di un Demux 2:1}
    \label{tab:demux_2:1}
\end{table}
Da cui si ricavano le seguenti equazioni relative alle uscite:
\begin{equation*}
    \begin{aligned}
          u_0 = (i_0 \text{ AND } \bar{s_0})\\
          u_1 = (i_0 \text{ AND } {s_0})
    \end{aligned}
    \end{equation*}
A partire dalla composizione di dispositivi di questo tipo, si può realizzare un Demultiplexer 1:4, come rappresentato in figura.
\begin{figure}[H]
	\centering
	\includegraphics[width=0.8\textwidth]{img/demux_1_4}
	\caption{Demux 1:4 composto a partire da Demux 1:2}
	\label{Demux 1:2} 
\end{figure}
Utilizzando il Demultiplexer appena progettato, al cui ingresso si fa corrispondere l'uscita del Multiplexer 16:1, progettato nell'esercizio precedente, si ottiene la rete di interconnessione, così formata:
\begin{figure}[H]
	\centering
	\includegraphics[width=1\textwidth]{img/ReteInterconn_composta}
	\caption{Rete di interconnessione: funzionamento interno}
	\label{Demux 1:2} 
\end{figure}
Nell'immagine, i colori sono stati usati per rendere più chiari i collegamenti tra segnali.

\subsection{Implementazione}
Si inizia mostrando l'implementazione del Demultiplexer 1:2, fatta seguendo un'architettura di tipo Dataflow.
\begin{code}
    \inputminted[frame=lines, framesep=2mm, baselinestretch=1.2, bgcolor=LightGray, fontsize=\footnotesize, linenos]{vhdl}{vhdl_files/demux1_2.vhd}
    \caption{Demultiplexer 1:2}
    \label{lst: demux_1_2}
\end{code}
Come mostrato dalla figura 1.12 presente nella fase di progettazione, a partire da 3 demux 1:2 si può realizzare un demux 1:4 seguendo un approccio di tipo strutturale. Segue il codice:
\begin{code}
    \inputminted[frame=lines, framesep=2mm, baselinestretch=1.2, bgcolor=LightGray, fontsize=\footnotesize, linenos]{vhdl}{vhdl_files/demux1_4.vhd}
    \caption{Demultiplexer 1:4}
    \label{lst: demux_1_4}
\end{code}
Tramite un'appropriata connessione del Multiplexer realizzato nell'esercizio precedente e il Demux 1:4, si ottiene la rete di interconnessione richiesta:
\begin{code}
    \inputminted[frame=lines, framesep=2mm, baselinestretch=1.2, bgcolor=LightGray, fontsize=\footnotesize, linenos]{vhdl}{vhdl_files/interc16_4.vhd}
    \caption{Rete di interconnessione 16:4 in VHDL}
    \label{lst: R_int16_4}
\end{code}
La rete realizzata è osservabile come schematic generato da Vivado:
\begin{figure}[H]
	\centering
	\includegraphics[width=1\textwidth]{img/reteIntVivado}
	\caption{Rete di interconnessione: schematic}
	\label{R_int_schem} 
\end{figure}
\subsection{Simulazione}
Per procedere con la simulazione della rete realizzata, si utilizza un tesbench. Tale testebnch è stato realizzato tramite il sito Doulos, e sono stati manualmente aggiunti diversi casi di test:
\begin{code}
    \inputminted[frame=lines, framesep=2mm, baselinestretch=1.2, bgcolor=LightGray, fontsize=\footnotesize, linenos]{vhdl}{vhdl_files/intercTb16_4.vhd}
    \caption{Testbench: Rete di interconnessione 16:4}
    \label{lst: R_int16_4TB}
\end{code}
I risultati di tale simulazione sono osservabili nella seguente waveform realizzata dal tool di Vivado.
\begin{figure}[H]
	\centering
	\includegraphics[width=1\textwidth]{img/TbReteInt}
	\caption{Rete di interconnessione: waveform}
	\label{R_int_sim} 
\end{figure}

\section{Implementazione su board del punto precedente}
La board utilizzata è la \textbf{Nexys A7}, una scheda di sviluppo basata su FPGA progettata da Digilent. 
\begin{figure}[H]
	\centering
	\includegraphics[width=0.7\textwidth]{img/board}
	\caption{Board Nexys A7}
	\label{board} 
\end{figure}
\subsection{Traccia}
 Sintetizzare ed implementare su board il progetto della rete di interconnessione sviluppato al punto 1.2, utilizzando gli switch per fornire gli input di selezione e i led per visualizzare i 4 bit di uscita. Per quanto riguarda i 16 bit dato in input, essi devono essere immessi mediante switch, 8 bit alla volta, sviluppando un’apposita 
“rete di controllo” per l’acquisizione che utilizzi due bottoni della board per caricare rispettivamente la prima e la seconda metà del dato in ingresso.
\subsection{Implementazione}
Per permettere lo sviluppo sulla board, è stato necessario gestire gli input in modo appropriato; per fare ciò che viene richiesto, si è scelto di usare il bottone \textit{BTNL} per il caricamento della prima metà degli ingressi, il bottone \textit{BTNR} per il caricamento della seconda metà degli ingressi, e il bottone \textit{BTNU} per il caricamento dei segnali di selezione; inoltre è stato previsto un bottone per il reset, \textit{BTNC}. Gli ingressi sono stati gestiti con gli switch, e le uscite sono visiualizzabili tramite i led. I primi 8 switch (da 0 a 7) sono stati utilizzati per gli ingressi, mentre i successivi 6 (da 8 a 13) per le selezioni. I led utilizzati per le uscite sono invece i primi 4 (da 0 a 3).
Per permettere opportune connessioni tra i componenti hardware e i segnali utilizzati nella rete di interconnessione, è stata implementata una unità di controllo, che ha gestito gli ingressi in due fasi distinte, oltre che i segnali di selezione.
Segue il codice dell'unità di controllo:
\begin{code}
    \inputminted[frame=lines, framesep=2mm, baselinestretch=1.2, bgcolor=LightGray, fontsize=\footnotesize, linenos]{vhdl}{vhdl_files/control_unit.vhd}
    \caption{Control unit}
    \label{lst: c_unit}
\end{code}
Inoltre, per consentire il funzionamento del sistema sulla board, è stato implementato il seguente codice:
\begin{code}
    \inputminted[frame=lines, framesep=2mm, baselinestretch=1.2, bgcolor=LightGray, fontsize=\footnotesize, linenos]{vhdl}{vhdl_files/interc_16_4onBoard.vhd}
    \caption{Implementazione: Rete di interconnessione on Board}
    \label{lst: R_i_onBoard}
\end{code}
\subsection{Funzionamento}
Di seguito si mostra l'esecuzione su board di uno dei casi di test visti in precedenza nella fase di simulazione. In particolare è stato testato ciò che avveniva a 51 ns, e si può vedere che il led acceso corrisponde con l'uscita attesa \textit{y2}.
\begin{figure}[H]
	\centering
	\includegraphics[width=0.7\textwidth]{img/TestRI}
	\caption{Uscita y2 attiva}
	\label{testRI} 
\end{figure}


\chapter{Esercizio 2 - Sistema ROM+M}
Il sistema che si vuole costruire consiste in due elementi principali: una ROM (Read-Only-Memory) puramente combinatoria e una macchina combinatoria M, che esegue una trasformazione sui dati letti da M e li pone in uscita. La ROM si compone di 16 locazioni di memoria, ciascuna contenente una stringa di 8 bit. Il sistema prende in ingresso un indirizzo di 4 bit, che permetterà di accedere a una delle locazioni della ROM; il dato in tale locazione viene posto in uscita alla ROM, e quindi in ingresso alla macchina M. La macchina M deve effettuare una trasformazione sulla stringa di 8 bit, in modo da restituire in uscita una stringa di 4 bit. La trasformazione scelta consiste nel sommare i 4 bit più significativi della stringa con i 4 bit meno significativi, la stringa di 4 bit risultante sarà restituita come uscita all'intero sistema.
\begin{figure}[H]
	\centering
	\includegraphics[width=0.8\textwidth]{img/rom_M_black_box}
	\caption{ROM + M}
	\label{Rom_M} 
\end{figure}
\section{Progettazione}
La progettazione consiste nella realizzazione dei due componenti fondamentali del sistema: ROM e M. 
\section{Implementazione}
Dapprima si implementa la ROM, in cui sono memorizzati 16 elementi, ciascuno da 8 bit. Il codice sottostante crea l'entità ROM, al cui ingresso è presente un vettore da 4 bit di \texttt{std\_logic} che rappresenta l'indirizzo, e in uscita restituisce un vettore di 8 bit. Vengono poi definite le stringhe di bit contenute nella ROM. Nel processo \texttt{main}, si pone in uscita l'elemento corrispondente alla locazione \texttt{address}.\\
%\paragraph{Mux 2:1} Di seguito il codice riguardante il Mux 2:1.
\begin{code}
    \inputminted[frame=lines, framesep=2mm, baselinestretch=1.2, bgcolor=LightGray, fontsize=\footnotesize, linenos]{vhdl}{vhdl_files/ROM.vhd}
    \caption{Implementazione ROM in VHDL}
    \label{lst:ROM}
\end{code}
Si procede poi con l'implementazione del componente M, che effettua la trasformazione.
\begin{code}
    \inputminted[frame=lines, framesep=2mm, baselinestretch=1.2, bgcolor=LightGray, fontsize=\footnotesize, linenos]{vhdl}{vhdl_files/M.vhd}
    \caption{Macchina M}
    \label{lst:M}
\end{code}
Nel processo si pone come uscita della macchina la somma tra i bit più significativi dell'ingresso (dal bit 7 al 4) e dei bit meno significativi (dal bit 3 allo 0).\\
Le due componenti sono parte del sistema S che è così implementato:
\begin{code}
    \inputminted[frame=lines, framesep=2mm, baselinestretch=1.2, bgcolor=LightGray, fontsize=\footnotesize, linenos]{vhdl}{vhdl_files/ROMplusM.vhd}
    \caption{Sistema S}
    \label{lst:S}
\end{code}
Tale sistema è stato costruito come structural: sono stati dichiarati i componenti, e ne sono state definite le istanze. Si è utilizzato un segnale di supporto $u_0$, che funge da segnale intermedio tra l'uscita della ROM e l'ingresso della macchina. \\
Si osserva lo schematic fornito dall'ambiente di sviluppo Vivado:
\begin{figure}[H]
	\centering
	\includegraphics[width=1\textwidth]{img/ROM_plus_M}
	\caption{Schematic di S}
	\label{SchemS} 
\end{figure}

\section{Simulazione}
Per procedere alla simulazione si realizza un testbench, con diversi casi di test, che permettano di osservare il comportamento del sistema.
\begin{code}
    \inputminted[frame=lines, framesep=2mm, baselinestretch=1.2, bgcolor=LightGray, fontsize=\footnotesize, linenos]{vhdl}{vhdl_files/ROMplusM_tb.vhd}
    \caption{Testbench}
    \label{lst:S_TB}
\end{code}
La seguente figura  permette la visualizzazione delle waveform.
\begin{figure}[H]
	\centering
	\includegraphics[width=1\textwidth]{img/Sim_Rom_plus_M}
	\caption{Waveform della simulazione di S}
	\label{SchemS} 
\end{figure}
Si procede con dei test effettuati manualmente per mostrare la correttezza nel funzionamento del sistema S. Per consentire una maggiore leggibilità, si è scelto di visualizzare gli indirizzi come \texttt{Unsigned Decimal}.\\
Nel caso $A = 0$, si accede alla stringa $00001001$, sommando i bit meno significativi con quelli più significativi si ottiene $0000+1001$ = $1001$;
nel caso $A = 5$, si accede alla stringa $01001010$, e procedendo come sopra si ottiene $0100+1010$ = $1110$.\\
Come si può vedere, i risultati di questi test coincidono con il comportamento atteso dal sistema e che sono mostrati nella waveform relativa alla simulazione. 

\section{Implementazione su board}
\subsection{Traccia}
 Sintetizzare ed implementare su board il progetto del sistema ROM+M sviluppato al punto 2.1, utilizzando gli switch per fornire l'indirizzo della ROM da cui leggere i valori da trasformare e i led per visualizzare i 4 bit di uscita.

\subsection{Implementazione}
In questo caso, per implementare il sistema sulla board, è stato sufficiente modificare il file \texttt{Nexys-A7-50T-Master.xdc}, collegando i primi 4 switch (da 0 a 3) all'indirizzo \textit{A} in ingresso, e i led da 0 a 3 alle uscite bout della macchina.\\
In particolare, il file xdc è composto dalle seguenti righe utili:
{\footnotesize
\begin{verbatim}
#GESTIONE SWITCH
set_property -dict {PACKAGE_PIN J15 IOSTANDARD LVCMOS33} [get_ports{ A[0]}]; 
# IO_L24N_T3_RS0_15 Sch=sw[0]
set_property -dict {PACKAGE_PIN L16 IOSTANDARD LVCMOS33} [get_ports{ A[1]}]; 
# IO_L3N_T0_DQS_EMCCLK_14 Sch=sw[1]
set_property -dict {PACKAGE_PIN M13 IOSTANDARD LVCMOS33} [get_ports{ A[2]}]; 
# IO_L6N_T0_D08_VREF_14 Sch=sw[2]
set_property -dict {PACKAGE_PIN R15 IOSTANDARD LVCMOS33} [get_ports{ A[3]}]; 
# IO_L13N_T2_MRCC_14 Sch=sw[3]
#GESTIONE LED
set_property -dict {PACKAGE_PIN H17 IOSTANDARD LVCMOS33} [get_ports {bout[0]}]; 
#IO_L18P_T2_A24_15 Sch=led[0]
set_property -dict {PACKAGE_PIN K15 IOSTANDARD LVCMOS33} [get_ports {bout[1]}]; 
#IO_L24P_T3_RS1_15 Sch=led[1]
set_property -dict {PACKAGE_PIN J13 IOSTANDARD LVCMOS33} [get_ports {bout[2]}]; 
#IO_L17N_T2_A25_15 Sch=led[2]
set_property -dict {PACKAGE_PIN N14 IOSTANDARD LVCMOS33} [get_ports {bout[3]}]; 
#IO_L8P_T1_D11_14 Sch=led[3]
\end{verbatim}
}

Si mostrano in seguito alcuni test eseguiti sulla board, che hanno confermato i risultati ottenuti dalla simulazione.
\begin{figure}[H]
	\centering
	\includegraphics[width=0.7\textwidth]{img/testRom_M_1}
	\caption{A = "0101", bout = "1110"}
	\label{test1} 
\end{figure}

\begin{figure}[H]
	\centering
	\includegraphics[width=0.7\textwidth]{img/testRom_M_2}
	\caption{A = "0001", bout = "0010"}
	\label{SchemS} 
\end{figure}

\begin{figure}[H]
	\centering
	\includegraphics[width=0.7\textwidth]{img/testrom_M_3}
	\caption{A = "1001", bout = "1001"}
	\label{SchemS} 
\end{figure}
Confrontando i risultati ottenuti con la waveform generata dalla simulazione si confermano le conclusioni precedenti.




\chapter{Esercizio 3}
\section{Riconoscitore di sequenze}
Un \textbf{riconoscitore di sequenze}, è una macchina sequenziale impulsiva\footnote{Macchina in cui l'uscita è vera solo per un determinato stato e per un determinato ingresso, e poi torna ad essere falsa.} che riceve una sequenza di bit in ingresso e che, a seconda se tale sequenza sia uguale o non ad una data, ritorni i valori 1 e 0, rispettivamente.\\
In particolare si possono avere due tipi di riconoscitori:
\begin{enumerate}
    \item \textbf{riconoscitori di sequenze non sovrapposte}: valuta i bit in ingresso a gruppi di $n$ elementi alla volta;
    \item \textbf{riconoscitori di sequenze parzialmente sovrapposte}: valuta i bit in ingresso a uno alla volta, tornando allo stato iniziale ogni qual volta la sequenza viene riconosciuta.
\end{enumerate}

\noindent Nel caso in esame si vuole implementare un riconoscitore della sequenza $\mathbf{101}$.\\
Oltre al dato, tale macchina ha in ingresso la tempificazione $A$ e il valore $M$, che nel caso in cui $M=0$, la macchina lavora come riconoscitore di sequenze non sovrapposte, mentre se $M=1$ lavora come  riconoscitore di sequenze parzialmente sovrapposte.

\subsection{Progettazione e architettura}
Per progettare una macchina sequenziale, vi è bisogno dell'automa a stati finiti.\\
Nel caso in questione, vi è il seguente risultato
\begin{figure}[H]
	\centering
	\includegraphics[width=0.7\textwidth]{img/automa_riconoscitore}
	\caption{Automa riconoscitore di sequenza}
	\label{aut_ric_seq} 
\end{figure}

\subsection{Implementazione}
Per l'implementazione VHDL dell'automa, si dichiarano dapprima gli ingressi
\begin{itemize}
    \item \texttt{RST}: permette il reset della macchina, portandola allo stato $S_0$;
    \item \texttt{A}: rappresenta l'abilitazione, ovvero il clock;
    \item \texttt{i}: è l'ingresso;
    \item \texttt{M}: permette di selezionare con quale modalità far lavorare la macchina: se $M=0$ effettua il riconoscimento a gruppi di tre bit per volta; se $M=1$ effettua il riconoscimento un bit alla volta
\end{itemize}
L'uscita è rappresentata dal segnale \texttt{Y}.\\
L'architettura è costruita con un approccio comportamentale e vi è una variazione di stato ad ogni fronte di salita del clock (\texttt{A}).\\
Si vuole notare che il segnale \texttt{RST} è sincrono.

\begin{code}
    \inputminted[frame=lines, framesep=2mm, baselinestretch=1.2, bgcolor=LightGray, fontsize=\footnotesize, linenos]{vhdl}{vhdl_files/riconoscitore.vhdl}
    \caption{riconoscitore.vhdl}
    \label{lst:RIC}
\end{code}

\subsection{Simulazione}
Per effettuare la simulazione, è stato necessario il seguente testbench.
\begin{code}
    \inputminted[frame=lines, framesep=2mm, baselinestretch=1.2, bgcolor=LightGray, fontsize=\footnotesize, linenos]{vhdl}{vhdl_files/riconoscitore_tb.vhdl}
    \caption{riconoscitore\_tb.vhdl}
    \label{lst:RIC}
\end{code}

Gli ingressi sono i seguenti:
\begin{itemize}
    \item $M=1$:
        \begin{itemize}
            \item 1, 1, 0, 1, 0, 0, 1, 0, 1
        \end{itemize}
    \item $M=0$:
        \begin{itemize}
            \item 1, 1, 0, 1, 0, 1, 1, 0, 1
        \end{itemize}
\end{itemize}

Il risultato è il seguente:

\begin{figure}[H]
	\centering
	\includegraphics[width=1\textwidth]{img/Sim_RIC}
	\caption{Simulazione Riconosciore}
	\label{aut_ric_seq} 
\end{figure}

%\begin{itemize}
%    \item \textbf{Stato S0}:
%        \begin{itemize}
%            \item se $i=0$, si rimane in \textbf{S0};
%            \item se $i=1$, si raggiunge lo stato \textbf{S1};
%        \end{itemize}
%    \item \textbf{Stato S1}:
%        \begin{itemize}
%            \item se $i=0$, si raggiunge lo stato \textbf{S2};
%            \item se $i=1$ e $M=0$ si rimane nello stato \textbf{S1}
%        \end{itemize}
%    \item \textbf{Stato S2}:
%        \begin{itemize}
%            \item se $i=0$ e $M=1$, si raggiunge lo stato \textbf{S0}
%            \item se $i=1$ si raggiunge lo stato \textbf{S3}
%        \end{itemize}
%    \item \textbf{Stato S3}:
%        \begin{itemize}
%            \item se $M=1$, si raggiunge lo stato \textbf{S0} con uscita $Y=1$
%            \item se $M=0$, si raggiunge lo stato \textbf{S1} con uscita $Y=1$
%        \end{itemize}
%\end{itemize}

\section{Implementazione su board del punto precedente}
\subsection{Traccia}
 Sintetizzare e implementare su board la rete sviluppata al punto precedente, utilizzando uno switch S1 per codificare l’input i e uno switch S2 per codificare il modo M, in combinazione con due bottoni B1 e B2 utilizzati rispettivamente per acquisire l’input da S1 e S2 in sincronismo con il segnale di tempificazione A, che 
deve essere ottenuto a partire dal clock della board. Infine, l’uscita Y può essere codificata utilizzando un led.
\subsection{Implementazione}
Il Riconoscitore di sequenza viene ripreso dal punto precedente, di conseguenza il suo codice viene importato nel progetto senza variazioni.\\
Per gestire il funzionamento di tale sistema su board, prima di tutto si rende necessario l'utilizzo di un divisore di frequenze che ha la funzione di generare un segnale di clock con una frequenza più bassa rispetto al clock di ingresso. In particolare il  processo implementato genera un clock di uscita con una frequenza pari a quella di ingresso divisa per il valore di \textit{DIVISOR}.
Si mostra il codice:
\begin{code}
    \inputminted[frame=lines, framesep=2mm, baselinestretch=1.2, bgcolor=LightGray, fontsize=\footnotesize, linenos]{vhdl}{vhdl_files/seq_rec_Board/freq_div.vhd}
    \caption{frequency\_divider.vhdl}
    \label{lst:SR_beh}
\end{code}
Per la gestione degli input tramite i bottoni della board, è stata realizata una unità di controllo; si usa lo $switch[0]$ unito al bottone \textit{BTNL} per l'ingresso "i", e lo $switch[1]$ unito al bottone \textit{BTNR} per l'ingresso "M". Si è scelto inoltre di mostrare le variazioni del clock sul $led[0]$, in modo da poter inserire correttamente gli input in corrispondenza del fronte di salita. L'uscita viene invece visualizzata sul $led[1]$.
\begin{code}
    \inputminted[frame=lines, framesep=2mm, baselinestretch=1.2, bgcolor=LightGray, fontsize=\footnotesize, linenos]{vhdl}{vhdl_files/seq_rec_Board/control_unit.vhd}
    \caption{control\_unit.vhdl}
    \label{lst:SR_beh}
\end{code}
Il codice del sistema su board, nel suo complesso, è stato realizzato seguendo un approccio di tipo strutturale:
\begin{code}
    \inputminted[frame=lines, framesep=2mm, baselinestretch=1.2, bgcolor=LightGray, fontsize=\footnotesize, linenos]{vhdl}{vhdl_files/seq_rec_Board/RecOnBoard.vhd}
    \caption{Riconoscitore su board in vhdl}
    \label{lst:SR_beh}
\end{code}


\chapter{Esercizio 4}
\section{Shift Register - Approccio comportamentale}
Si vuole implementare uno Shift Register con approccio comportamentale, la cui dimensione è $N$.\\
Tale macchina ha come ingresso un valore $y$ con il quale si può scegliere di fare shift verso destra o sinistra e di fare shift di uno o due bit.
\subsection{Progetto e architettura}
La macchina da implementare è la seguente:
\begin{figure}[H]
	\centering
	\includegraphics[width=0.7\textwidth]{img/shift_register}
	\caption{Shift Register}
	\label{shf_reg} 
\end{figure}
Dall'immagine si può notare che è stato scelto $N=8$.\\
Gli ingressi sono i seguenti:
\begin{itemize}
    \item \texttt{clk}: il clock, necessario per la tempificazione. La macchina lavorerà sul fronte di salita;
    \item \texttt{en}: l'abilitazione, la quale permette di abilitare o disabilitare la macchina;
    \item \texttt{rst}: reset sincrono della macchina;
    \item \texttt{y}: vettore di 2 elementi che sceglie la modalità di funzionamento della macchina; in particolare:
        \begin{itemize}
            \item $y=00$: shift a sinistra di 1;
            \item $y=01$: shift a sinistra di 2;
            \item $y=10$: shift a destra di 1;
            \item $y=11$: shift a destra di 2;
        \end{itemize}
    \item  \texttt{data\_in}: rappresenta i dati in ingresso; esso è un vettore di due elementi poiché quando vi è la necessità di fare uno shift di 2, si ha bisogno di due bit
\end{itemize}
L'uscita della macchina è \texttt{data\_out}, vettore di 8 bit.
\subsection{Implementazione}
L'implementazione è la seguente
\begin{code}
    \inputminted[frame=lines, framesep=2mm, baselinestretch=1.2, bgcolor=LightGray, fontsize=\footnotesize, linenos]{vhdl}{vhdl_files/shift_registrer_beh.vhdl}
    \caption{shift\_register\_beh.vhdl}
    \label{lst:SR_beh}
\end{code}

\subsection{Simulazione}
Per effettuare la simulazione, si utilizza il seguente testbench:
\begin{code}
    \inputminted[frame=lines, framesep=2mm, baselinestretch=1.2, bgcolor=LightGray, fontsize=\footnotesize, linenos]{vhdl}{vhdl_files/shift_register_beh_tb.vhdl}
    \caption{shift\_register\_beh\_tb.vhdl}
    \label{lst:SR_beh_tb}
\end{code}
\noindent Il risultato della simulazione è il seguente:

\begin{figure}[H]
	\centering
	\includegraphics[width=1\textwidth]{img/simulazione_shift_register.png}
	\caption{Simulazione Shift Register con approccio comportamentale}
	\label{shf_reg_beh_sim} 
\end{figure}
Si può facilmente notare dall'immagine che la macchina lavora come desiderato: ad ogni fronte di salita del clock e quando l'abilitazione è alta, in base alla modalità di lavoro, shifta a destra o a sinistra, di uno o due bit.

\section{Shift Register - Approccio strutturale}
Si vuole riprogettare la macchina precedente, figura \ref{shf_reg}, utilizzando un approccio strutturale.\\
Le componenti della macchina sono 8 registri da 1 bit e 8 mux 4:1.\\
Si sono scelti 8 registri poiché in tale esempio si realizza un registro da 8 bit ($N=8$).
\subsection{Progetto e architettura}
\subsubsection{Registro da un bit}
Il primo componente necessario è il registro da un bit.
\begin{figure}[H]
	\centering
	\includegraphics[width=0.7\textwidth]{img/reg.png}
	\caption{Registro da 1 bit}
	\label{shf_reg_beh_sim} 
\end{figure}
Gli ingressi di tale componente sono i seguenti:
\begin{itemize}
    \item \texttt{data\_in}: bit in ingresso, che verrà memorizzato nel registro;
    \item \texttt{clk}: il clock per la tempificazione; il registro lavora sul fronte di salita di quest'ultimo;
    \item \texttt{en}: segnale di abilitazione; il registro memorizza il bit in ingresso solo quando tale segnale è alto;
    \item \texttt{rst}: segnale che quando è alto resetta il registro, portando il valore al suo interno a 0; il reset è sincrono.
\end{itemize}
La sua uscita è \texttt{data\_out}, che altro non rappresenta il bit memorizzato nel registro.
\subsubsection{Mux 4:1}
Il secondo componente è il mux 4:1.\\
Tramite quest'ultimo si decide qual è l'ingresso di un registro, attraverso la selezione.\\
\begin{figure}[H]
	\centering
	\includegraphics[width=0.5\textwidth]{/mux_41_sr.png}
	\caption{Mux 4:1}
	\label{mux_41_sr} 
\end{figure}
\noindent Tale multiplexer lavora seguendo la seguente tabella:
\begin{table}[h!]
    \centering
    \begin{tabular}{||c|c||c||}
        \hline
        \hline
        $\mathbf{y_1}$ & $\mathbf{y_0}$ & $\mathbf{u}$ \\
        \hline
        0 & 0 & $\mathbf{i_0}$ \\
        \hline
        0 & 1 & $\mathbf{i_1}$ \\
        \hline
        1 & 0 & $\mathbf{i_2}$ \\
        \hline
        1 & 1 & $\mathbf{i_3}$ \\
        \hline
        \hline 
    \end{tabular}
    \caption{Tabella di verità del multiplexer 4:1 per lo Shift Register}
    \label{tab:mux_4:1_sr}
\end{table}
Si può banalmente notare che l'uscita altro non è che uno dei 4 ingressi del multiplexer, scelto tramite la selezione.

Si può ora comporre lo Shift Register.\\
Facendo gli opportuni collegamenti, si ottiene il seguente schema
\begin{figure}[H]
	\centering
	\includegraphics[width=1\textwidth]{shiftregister_str.png}
	\caption{Shift Register con approccio strutturale}
	\label{mux_41_sr} 
\end{figure}
Gli ingressi e l'uscita dello Shift Register sono identici a quelli visti nell'esercizio precedente.\\
Quello che si vuole mettere in evidenzia in questo caso è come tali ingressi siano collegati con le strutture interni: in particolare si vede che \texttt{y}, che sceglie la modalità di lavoro della macchina, è collegato alla selezione dei multiplexer, mentre \texttt{data\_in}, è collegato solo ai primi due e agli ultimi due multiplexer.\\
I restanti sono collegati direttamente ai registri.

\subsection{Implementazione}
Si vuole ora procedere con l'implementazione in VHDL.\\
Partendo dal registro, si ha:
\begin{code}
    \inputminted[frame=lines, framesep=2mm, baselinestretch=1.2, bgcolor=LightGray, fontsize=\footnotesize, linenos]{vhdl}{vhdl_files/Cap_4/register.vhdl}
    \caption{register.vhdl}
    \label{lst:rgt_one_bit}
\end{code}
\noindent Si prosegue con il multiplexer 4:1
\begin{code}
    \inputminted[frame=lines, framesep=2mm, baselinestretch=1.2, bgcolor=LightGray, fontsize=\footnotesize, linenos]{vhdl}{vhdl_files/Cap_4/mux_4_1.vhdl}
    \caption{mux\_4\_1.vhdl}
    \label{lst:mux_4_1_sr}
\end{code}
\noindent Implementate le componenti base per il progetto, si prosegue con lo Shift Register:
\begin{code}
    \inputminted[frame=lines, framesep=2mm, baselinestretch=1.2, bgcolor=LightGray, fontsize=\footnotesize, linenos]{vhdl}{vhdl_files/Cap_4/shift_register.vhdl}
    \caption{shift\_register.vhdl}
    \label{lst:sr_s}
\end{code}
Si può notare come nella architettura, sono state prima generate le componenti e solo dopo sono stati effettuati i vari collegamenti, utilizzando variabili ausiliarie.

\subsection{Simulazione}
Per effettuare la simulazione, si implementa il seguente testbench
\begin{code}
    \inputminted[frame=lines, framesep=2mm, baselinestretch=1.2, bgcolor=LightGray, fontsize=\footnotesize, linenos]{vhdl}{vhdl_files/Cap_4/tb_shift_register.vhdl}
    \caption{tb\_shift\_register.vhdl}
    \label{lst:tb_sr}
\end{code}
Lanciando la simulazione, il risultato è il seguente:
\begin{figure}[H]
	\centering
	\includegraphics[width=1\textwidth]{img/sim_shift_register_str.png}
	\caption{Simulazione dello Shift Register con approccio strutturale}
	\label{sim_sr_str} 
\end{figure}
Si nota chiaramente un corretto funzionamento della macchina.

\chapter{Esercizio 5}
\section{Cronometro}
Si vuole progettare, implementare e testare un cronometro, in grado di scandire secondi, minuti e ore, a partire da una base dei tempi prefissata (clock). \\
Si vuole inoltre che l'inizializzazione del cronometro possa essere fatta anche con un valore iniziale, espresso in ore, minuti e secondi, mediante un ingresso di \texttt{set}, e deve prevedere un ingresso di \texttt{reset} per azzerare il tempo.

\subsection{Progettazione}
Per la progettazione di tale macchina, si utilizza un approccio strutturale.\\
In linea generale, si parte da un flip-flop D per la composizione di contatori di modulo 64 e modulo 32. Questi ultimi saranno necessari per la progettazione di due contatori modulo 60 e un contatore modulo 24, rispettivamente.\\
Questi ultimi tre, collegati opportunamente, andranno a comporre il cronometro.

\subsubsection{Flip-Flop D}
Il Flip-Flop D è progettato come segue:
\begin{figure}[H]
	\centering
	\includegraphics[width=0.7\textwidth]{img/ffD}
	\caption{Flip-Flop D}
	\label{ffd} 
\end{figure}
La macchina lavora sul fronte di salita del clock e quando il valore \texttt{en} è alto. \\
Ha due uscite, la prima \texttt{output} presenta in uscita ciò che è memorizzato nel Flip-Flop, mentre \texttt{not\_output}, presenta il negato.

\subsubsection{Contatore modulo 64}
Il contatore modulo 64 che si vuole progettare è il seguente:
\begin{figure}[H]
	\centering
	\includegraphics[width=0.7\textwidth]{img/counter_64}
	\caption{Contatore modulo 64}
	\label{cnt_64} 
\end{figure}
Gli ingessi sono:
\begin{itemize}
    \item \texttt{clk}: clock per la tempificazione;
    \item \texttt{en}: segnale di abilitazione;
    \item \texttt{rst}: segnale di reset;
    \item \texttt{set} e \texttt{v\_set}: segnali necessari per il setting del valore di partenza
\end{itemize}
Le uscite sono \texttt{counter}, che altro non è il conteggio, e \texttt{count}, uscita necessaria per permettere il collegamento con altri contatori.\\

\noindent Si vuole progettare tale macchina strutturalmente, utilizzando come componente base il Flip-Flop D.\\
Per far ciò si collegano parallelamente cinque Flip-Flop D, nel modo seguente:
\clearpage
\begin{sidewaysfigure}[ht]
	\centering
	\includegraphics[width=1\textwidth]{img/counter_mod_64_str.png}
	\caption{Contatore modul0 64 - approccio strutturale}
	\label{cnt_64_str} 
\end{sidewaysfigure}
\clearpage

\subsubsection{Contatore modulo 32}
Il contatore modulo 32 è esteriormente identico a quello modulo 64:
\begin{figure}[H]
	\centering
	\includegraphics[width=0.7\textwidth]{img/counter_mod_32.png}
	\caption{Contatore modulo 32}
	\label{cnt_32} 
\end{figure}

\noindent Analogamente al caso precedente, si compone strutturalmente con i Flip-Flop D:
\clearpage
\begin{sidewaysfigure}[ht]
	\centering
	\includegraphics[width=1\textwidth]{img/counter_mod_32_str.png}
	\caption{Contatore modulo 32 - approccio strutturale}
	\label{cnt_32_str} 
\end{sidewaysfigure}
\clearpage

\subsubsection{Contatore modulo 60}
Per progettare un contatore modulo 60, si parte da un contatore modulo 64 facendo in modo che ogni qual volta raggiunga 59 si resetti.
\begin{figure}[H]
	\centering
	\includegraphics[width=1\textwidth]{img/counter_mod_60.png}
	\caption{Contatore modulo 60}
	\label{cnt_60} 
\end{figure}

\subsubsection{Contatore modulo 24}
Analogamente al contatore precedente, si realizza il contatore modulo 24 utilizzando il contatore modulo 32 in modo che si resetti ogni qualvolta raggiunga 23
\begin{figure}[H]
	\centering
	\includegraphics[width=1\textwidth]{img/counter_mod_24.png}
	\caption{Contatore modulo 24}
	\label{cnt_60} 
\end{figure}

\subsubsection{Cronometro}
Si possono ora effettuare i collegamenti per creare un cronometro: ovviamente servono due contatori modulo 60 per secondi e minuti, e un contatore modulo 24:
\begin{figure}[H]
	\centering
	\includegraphics[width=1\textwidth]{img/stopwatch.png}
	\caption{Cronometro}
	\label{stpwth} 
\end{figure}

\subsection{Implementazione}
Si può a questo punto prcedere con l'implementazione in VHDL del nostro progetto, partendo dalle sue componenti.
\subsubsection{Flip-Flop D}
\begin{code}
    \inputminted[frame=lines, framesep=2mm, baselinestretch=1.2, bgcolor=LightGray, fontsize=\footnotesize, linenos]{vhdl}{vhdl_files/Esercizio_5.1/ffD.vhdl}
    \caption{ffD.vhdl}
    \label{lst:ffD}
\end{code}
Il Flip-Flop D è stato implementato con un approccio comportamentale e con reset sincrono.

\subsubsection{Contatore modulo 64}
Utilizzando il Flip-Flop D, si implementa ora il contatore mdoulo 64.
\begin{code}
    \inputminted[frame=lines, framesep=2mm, baselinestretch=1.2, bgcolor=LightGray, fontsize=\footnotesize, linenos]{vhdl}{vhdl_files/Esercizio_5.1/counter_mod_64.vhdl}
    \caption{counter\_mod\_64.vhdl}
    \label{lst:count_64}
\end{code}

\subsubsection{Contatore modulo 32}
In modo analogo, si procede per il contatore modulo 32.
\begin{code}
    \inputminted[frame=lines, framesep=2mm, baselinestretch=1.2, bgcolor=LightGray, fontsize=\footnotesize, linenos]{vhdl}{vhdl_files/Esercizio_5.1/counter_mod_32.vhdl}
    \caption{counter\_mod\_32.vhdl}
    \label{lst:count_32}
\end{code}

\subsubsection{Contatore modulo 60}
Si implemeta ora il contatore modulo 60, partendo da quello modulo 64.
\begin{code}
    \inputminted[frame=lines, framesep=2mm, baselinestretch=1.2, bgcolor=LightGray, fontsize=\footnotesize, linenos]{vhdl}{vhdl_files/Esercizio_5.1/counter_mod_60.vhdl}
    \caption{counter\_mod\_60.vhdl}
    \label{lst:count_60}
\end{code}

\subsubsection{Contatore modulo 24}
Analogamente al contatore modulo 60, si implementa il contatore modulo 24 a partire da quello di modulo 32.
\begin{code}
    \inputminted[frame=lines, framesep=2mm, baselinestretch=1.2, bgcolor=LightGray, fontsize=\footnotesize, linenos]{vhdl}{vhdl_files/Esercizio_5.1/counter_mod_24.vhdl}
    \caption{counter\_mod\_24.vhdl}
    \label{lst:count_24}
\end{code}

\subsubsection{Cronometro}
Si può finalmente implementare il cronometro, utilizzando gli ultimi due contatori implementati.
\begin{code}
    \inputminted[frame=lines, framesep=2mm, baselinestretch=1.2, bgcolor=LightGray, fontsize=\footnotesize, linenos]{vhdl}{vhdl_files/Esercizio_5.1/stopwatch.vhdl}
    \caption{stopwatch.vhdl}
    \label{lst:stopwatch}
\end{code}

\subsection{Simulazione}
Per la simulazione, si implementa il seguente testbench:
\begin{code}
    \inputminted[frame=lines, framesep=2mm, baselinestretch=1.2, bgcolor=LightGray, fontsize=\footnotesize, linenos]{vhdl}{vhdl_files/Esercizio_5.1/stopwatch_tb.vhdl}
    \caption{stopwatch\_tb.vhdl}
    \label{lst:stopwatch_tb}
\end{code}

Lanciando la simulazione si ha 
\begin{figure}[H]
	\centering
	\includegraphics[width=1\textwidth]{img/stopwatch_sim}
	\caption{Simulazione Cronometro}
	\label{stpwth_sim} 
\end{figure}
Si vuole far notare che per materiali, è impossibile riportare nella documentazione l'intera simulazione.\\
Per cui qualore si volesse testare il cronometro, il codice VHDL si trova nella repository GitHub associata al progetto, nella cartella \texttt{Esercizio 5.1}.

\section{Implementazione su board del punto precedente}
\subsection{Traccia}
 Sintetizzare ed implementare su board il componente sviluppato al punto precedente, utilizzando i display a 7 segmenti per la visualizzazione dell’orario (o una combinazione di display e led nel caso in cui i display a disposizione siano in numero inferiore a quello necessario), gli switch per l’immissione dell’orario iniziale e due bottoni, uno per il set dell’orario e uno per il reset. Si utilizzi una codifica a scelta dello studente per la visualizzazione dell’orario sui display (esadecimale o decimale).
 \subsection{Implementazione}
 Per l'implementazione su board è stato necessario utilizzare il visore presente sulla board, per fare ciò sono stati usati i seguenti codici:
 \begin{code}
    \inputminted[frame=lines, framesep=2mm, baselinestretch=1.2, bgcolor=LightGray, fontsize=\footnotesize, linenos]{vhdl}{vhdl_files/cron_board/anodes_manager.vhd}
    \caption{anodes\_manager.vhdl}
    \label{lbl:ROMC}
\end{code}
 \begin{code}
    \inputminted[frame=lines, framesep=2mm, baselinestretch=1.2, bgcolor=LightGray, fontsize=\footnotesize, linenos]{vhdl}{vhdl_files/cron_board/anodes_manager.vhd}
    \caption{catode\_manager.vhdl}
    \label{lbl:ROMC}
\end{code}
 \begin{code}
    \inputminted[frame=lines, framesep=2mm, baselinestretch=1.2, bgcolor=LightGray, fontsize=\footnotesize, linenos]{vhdl}{vhdl_files/cron_board/display_seven_segments.vhd}
    \caption{display\_seven\_segments.vhdl}
    \label{lbl:ROMC}
\end{code}
\begin{code}
    \inputminted[frame=lines, framesep=2mm, baselinestretch=1.2, bgcolor=LightGray, fontsize=\footnotesize, linenos]{vhdl}{vhdl_files/cron_board/clock_filter.vhd}
    \caption{clock\_filter.vhdl}
    \label{lbl:ROMC}
\end{code}
\begin{code}
    \inputminted[frame=lines, framesep=2mm, baselinestretch=1.2, bgcolor=LightGray, fontsize=\footnotesize, linenos]{vhdl}{vhdl_files/cron_board/count_mod8.vhd}
    \caption{count\_mod8.vhdl}
    \label{lbl:ROMC}
\end{code}
Inoltre per avere una rappresentazione coerente sui visori, sono stati implementati i seguenti codici per distinguere decine e unità dalle uscite del cronometro:
 \begin{code}
    \inputminted[frame=lines, framesep=2mm, baselinestretch=1.2, bgcolor=LightGray, fontsize=\footnotesize, linenos]{vhdl}{vhdl_files/cron_board/separator.vhd}
    \caption{separator.vhdl}
    \label{lbl:ROMC}
\end{code}
 \begin{code}
    \inputminted[frame=lines, framesep=2mm, baselinestretch=1.2, bgcolor=LightGray, fontsize=\footnotesize, linenos]{vhdl}{vhdl_files/cron_board/sep_h_m_s.vhd}
    \caption{anodes\_manager.vhdl}
    \label{lbl:ROMC}
\end{code}
Si è inoltre scelto di usare un divisore di frequenza per ottenere un clock più adatto al funzionamento del progetto:
\begin{code}
    \inputminted[frame=lines, framesep=2mm, baselinestretch=1.2, bgcolor=LightGray, fontsize=\footnotesize, linenos]{vhdl}{vhdl_files/cron_board/freq_divider.vhd}
    \caption{freq\_divider.vhdl}
    \label{lbl:ROMC}
\end{code}
Per la gestione degli input, soprattutto relativi al \textit{SET} di secondi, minuti e ore, si è implementata una unotà di controllo, che sulla base del bottone premuto caricherà rispettivamente secondi (BTNU), minuti (BTNL) e ore (BTNR), e con un ulteriore bottone viene abilitato il SET (BTND). Inoltre con il bottone centrale si abilita il RESET.
\begin{code}
    \inputminted[frame=lines, framesep=2mm, baselinestretch=1.2, bgcolor=LightGray, fontsize=\footnotesize, linenos]{vhdl}{vhdl_files/cron_board/control_unit.vhd}
    \caption{control\_unit.vhdl}
    \label{lbl:ROMC}
\end{code}
La gestione del sistema nel suo complesso è implementata con il seguente codice:
\begin{code}
    \inputminted[frame=lines, framesep=2mm, baselinestretch=1.2, bgcolor=LightGray, fontsize=\footnotesize, linenos]{vhdl}{vhdl_files/cron_board/cron_onBOARD.vhd}
    \caption{cron\_onBoard.vhdl}
    \label{lbl:ROMC}
\end{code}
Si mostra anche il codice di Nexys-A7-50T-Master.xdc, fondamentale per la generazione del bitstream e di conseguenza per la programmazione della board.
{\footnotesize
\begin{verbatim}
## Clock signal
set_property -dict {PACKAGE_PIN E3 IOSTANDARD LVCMOS33}
[get_ports {CLK}]; #IO_L12P_T1_MRCC_35 Sch=clk100mhz
create_clock -add -name sys_clk_pin -period 10.00 -waveform {0 5} 
[get_ports {CLK}];

##Switches
set_property -dict {PACKAGE_PIN J15   IOSTANDARD LVCMOS33} 
[get_ports {value_in[0]}]; #IO_L24N_T3_RS0_15 Sch=sw[0]
set_property -dict { PACKAGE_PIN L16   IOSTANDARD LVCMOS33 } 
[get_ports { value_in[1] }]; #IO_L3N_T0_DQS_EMCCLK_14 Sch=sw[1]
set_property -dict { PACKAGE_PIN M13   IOSTANDARD LVCMOS33 } 
[get_ports { value_in[2] }]; #IO_L6N_T0_D08_VREF_14 Sch=sw[2]
set_property -dict { PACKAGE_PIN R15   IOSTANDARD LVCMOS33 } 
[get_ports { value_in[3] }]; #IO_L13N_T2_MRCC_14 Sch=sw[3]
set_property -dict { PACKAGE_PIN R17   IOSTANDARD LVCMOS33 } 
[get_ports { value_in[4] }]; #IO_L12N_T1_MRCC_14 Sch=sw[4]
set_property -dict { PACKAGE_PIN T18   IOSTANDARD LVCMOS33 } 
[get_ports { value_in[5] }]; #IO_L7N_T1_D10_14 Sch=sw[5]

##7 segment display
set_property -dict { PACKAGE_PIN T10   IOSTANDARD LVCMOS33 } 
[get_ports { cathodes_out[0] }]; #IO_L24N_T3_A00_D16_14 Sch=ca
set_property -dict { PACKAGE_PIN R10   IOSTANDARD LVCMOS33 } 
[get_ports { cathodes_out[1] }]; #IO_25_14 Sch=cb
set_property -dict { PACKAGE_PIN K16   IOSTANDARD LVCMOS33 } 
[get_ports { cathodes_out[2] }]; #IO_25_15 Sch=cc
set_property -dict { PACKAGE_PIN K13   IOSTANDARD LVCMOS33 } 
[get_ports { cathodes_out[3] }]; #IO_L17P_T2_A26_15 Sch=cd
set_property -dict { PACKAGE_PIN P15   IOSTANDARD LVCMOS33 } 
[get_ports { cathodes_out[4] }]; #IO_L13P_T2_MRCC_14 Sch=ce
set_property -dict { PACKAGE_PIN T11   IOSTANDARD LVCMOS33 } 
[get_ports { cathodes_out[5] }]; #IO_L19P_T3_A10_D26_14 Sch=cf
set_property -dict { PACKAGE_PIN L18   IOSTANDARD LVCMOS33 } 
[get_ports { cathodes_out[6] }]; #IO_L4P_T0_D04_14 Sch=cg
set_property -dict { PACKAGE_PIN H15   IOSTANDARD LVCMOS33 } 
[get_ports { cathodes_out[7] }]; #IO_L19N_T3_A21_VREF_15 Sch=dp
set_property -dict { PACKAGE_PIN J17   IOSTANDARD LVCMOS33 } 
[get_ports { anodes_out[0] }]; #IO_L23P_T3_FOE_B_15 Sch=an[0]
set_property -dict { PACKAGE_PIN J18   IOSTANDARD LVCMOS33 } 
[get_ports { anodes_out[1] }]; #IO_L23N_T3_FWE_B_15 Sch=an[1]
set_property -dict { PACKAGE_PIN T9    IOSTANDARD LVCMOS33 } 
[get_ports { anodes_out[2] }]; #IO_L24P_T3_A01_D17_14 Sch=an[2]
set_property -dict { PACKAGE_PIN J14   IOSTANDARD LVCMOS33 } 
[get_ports { anodes_out[3] }]; #IO_L19P_T3_A22_15 Sch=an[3]
set_property -dict { PACKAGE_PIN P14   IOSTANDARD LVCMOS33 } 
[get_ports { anodes_out[4] }]; #IO_L8N_T1_D12_14 Sch=an[4]
set_property -dict { PACKAGE_PIN T14   IOSTANDARD LVCMOS33 } 
[get_ports { anodes_out[5] }]; #IO_L14P_T2_SRCC_14 Sch=an[5]
set_property -dict { PACKAGE_PIN K2    IOSTANDARD LVCMOS33 } 
[get_ports { anodes_out[6] }]; #IO_L23P_T3_35 Sch=an[6]
set_property -dict { PACKAGE_PIN U13   IOSTANDARD LVCMOS33 } 
[get_ports { anodes_out[7] }]; #IO_L23N_T3_A02_D18_14 Sch=an[7]

##Buttons
#set_property -dict { PACKAGE_PIN C12   IOSTANDARD LVCMOS33 } 
[get_ports { reset_in }]; #IO_L3P_T0_DQS_AD1P_15 Sch=cpu_resetn
set_property -dict { PACKAGE_PIN N17   IOSTANDARD LVCMOS33 } 
[get_ports { RST }]; #IO_L9P_T1_DQS_14 Sch=btnc
set_property -dict { PACKAGE_PIN M18   IOSTANDARD LVCMOS33 } 
[get_ports { load_s }]; #IO_L4N_T0_D05_14 Sch=btnu
set_property -dict { PACKAGE_PIN P17   IOSTANDARD LVCMOS33 } 
[get_ports { load_m }]; #IO_L12P_T1_MRCC_14 Sch=btnl
set_property -dict { PACKAGE_PIN M17   IOSTANDARD LVCMOS33 } 
[get_ports { load_h }]; #IO_L10N_T1_D15_14 Sch=btnr
set_property -dict { PACKAGE_PIN P18   IOSTANDARD LVCMOS33 } 
[get_ports { SET }]; #IO_L9N_T1_DQS_D13_14 Sch=btnd


\end{verbatim}
}


\chapter{Esercizio 6}
\section{Sistema di lettura - elaborazione - scrittura PO\_PC}
\subsection{Traccia}
 Progettare, implementare in VHDL e verificare mediante simulazione un sistema dotato di una memoria ROM di N locazioni da 8 bit ciascuna, una macchina combinatoria M in grado di trasformare (secondo una funzione a scelta dello studente) la stringa di 8 bit letta dalla ROM in una stringa di 4 bit, e una memoria MEM di N locazioni che memorizza la stringa in output da M. Il sistema si avvia in corrispondenza di un segnale di START che viene fornito esternamente. Una volta avviato, tramite un’apposita unità di controllo che gestisce la tempificazione del sistema, viene scandita una locazione alla volta della ROM e viene scritta la corrispondente locazione di MEM. Gli indirizzi di memoria sono forniti da un contatore. Le memorie ROM e MEM hanno rispettivamente un read e un write sincrono.
\subsection{Progettazione}
Per il progetto di questo sistema si riprende l'esercizio nel capitolo 2: \textbf{Sistema ROM + M}. Viene usato anche un contatore, per scandire una alla volta tutte le locazioni della ROM da cui prelevare la stringa contenente 8 bit. Come nell'esercizio precedente, l'uscita della ROM viene posta in ingresso alla macchina M, che somma i 4 bit più significativi dell'ingresso con i 4 bit meno significativi. L'uscita della macchina M viene posto in ingresso a una memoria MEM e poi caricato nella locazione corrispondente all'uscita del contatore; il funzionamento del sistema viene gestito da un'unità di controllo.  
La struttura del sistema sarà fatta in questo modo:
\begin{figure}[H]
	\centering
	\includegraphics[width=1\textwidth]{img/ROM_M_MEM.png}
	\caption{Schema a blocchi del sistema ROM + M + MEM}
	\label{shf_reg_beh_sim} 
\end{figure}
L'unità di controllo (CU) può essere efficacemente modellata come una macchina a stati finiti (FSM). Nel caso particolare avremo i seguenti stati:
\begin{itemize}
    \item \textbf{idle}
    \item \textbf{read}
    \item \textbf{m\_work}
    \item \textbf{write}
\end{itemize}
L'unità di controllo può essere quindi rappresentata da un automa come si mostra in figura:
\begin{figure}[H]
	\centering
	\includegraphics[width=1\textwidth]{img/Automa_ROM_M_MEM.png}
	\caption{Macchina a stati della control unit di ROM + M + MEM}
	\label{automa_rom_m_mem} 
\end{figure}
\subsection{Implementazione}
Il codice implementativo di M resta invariato, in quanto macchina puramente combinatoria. Si nota che viene richiesto che le operazioni di Read dalla ROM e di Write sulla memoria MEM siano svolte in modo sincrono. Quindi, a differenza della ROM usata nell'esercizio precedente, che era puramente combinatoria, le operazioni di lettura di questa ROM avvengono in sincronia con un segnale di clock. Questo segnale fornisce un riferimento temporale preciso per tutte le operazioni interne della ROM, garantendo così un funzionamento coerente e affidabile. Inoltre, sono stati utilizzati dei segnali di abilitazione alla lettura e alla scrittura, in modo da evitare conflitti e da permettere che i dati vengano letti al momento giusto. 
Si mostra innanzitutto il nuovo codice di ROM:
\begin{code}
    \inputminted[frame=lines, framesep=2mm, baselinestretch=1.2, bgcolor=LightGray, fontsize=\footnotesize, linenos]{vhdl}{vhdl_files/ROM_MEM_M/ROMconCLOCK.vhd}
    \caption{ROM.vhdl}
    \label{lbl:ROMC}
\end{code}
Si è scelto di utilizzare le stesse stringhe dell'esercizio 2 per "popolare" la ROM, in modo da poter confrontare i risultati.
Si mostra ora l'implementazione dell'unità di controllo del sistema; tale codice gestisce i cambiamenti di stato, e si può considerare il "cervello" del sistema in esame. \\
Si noti che si è scelto di porre in uscita gli stati, in modo da visualizzare in simulazione anche le variazioni di stato, è una scelta ovviamente facoltativa, ma per ragione di debugging è stato scelto di visualizzare anche la variazione di stato, come sarà visibile dalla waveform nella prossima sezione.
\begin{code}
    \inputminted[frame=lines, framesep=2mm, baselinestretch=1.2, bgcolor=LightGray, fontsize=\footnotesize, linenos]{vhdl}{vhdl_files/ROM_MEM_M/control_unit.vhd}
    \caption{control unit.vhdl}
    \label{lbl:ROMC}
\end{code}
Si mostra infine il codice sistema nel suo complesso, composto dall'unità di controllo e da tutte le altre componenti utilizzate; è stato utilizzato un approccio di tipo strutturale:
\begin{code}
    \inputminted[frame=lines, framesep=2mm, baselinestretch=1.2, bgcolor=LightGray, fontsize=\footnotesize, linenos]{vhdl}{vhdl_files/ROM_MEM_M/ROM_M_MEMO.vhd}
    \caption{ROM + M + MEM.vhdl}
    \label{lbl:ROMC}
\end{code}
Si vuole porre l'attenzione allo Schematic generato dall'ambiente Vivado, che mostra chiaramente i collegamenti e le dipendenze tra tutte le componenti del sistema.
\begin{figure}[H]
	\centering
	\includegraphics[width=1\textwidth]{img/schematic_ROM_M_MEM.PNG}
	\caption{Schematic di ROM + M + MEM}
	\label{schemROM_M_MEM} 
\end{figure}
\subsection{Simulazione}
Per procedere con la simulazione è stato necessario generare un testbench:
\begin{code}
    \inputminted[frame=lines, framesep=2mm, baselinestretch=1.2, bgcolor=LightGray, fontsize=\footnotesize, linenos]{vhdl}{vhdl_files/ROM_MEM_M/Rom_M_MEM_tb.vhd}
    \caption{Testbench di ROM + M + MEM.vhdl}
    \label{lbl:ROMC}
\end{code}
Eseguendo la simulazione si avrà la seguente forma d'onda:
\begin{figure}[H]
	\centering
	\includegraphics[width=1\textwidth]{img/ROM_M_MEM_waweform.PNG}
	\caption{Waveform di ROM + M + MEM}
	\label{schemROM_M_MEM} 
\end{figure}
Si analizzano alcuni casi per dimostrare la correttezza del sistema.
\begin{itemize}
    \item \textbf{istante 0:} si accede alla locazione di memoria $0$, in cui si trova la stringa $"00001001"$, sommando i 4 bit più significativi con i 4 meno significativi si ottiene $1001$;
    \item \textbf{istante 1:} si accede alla locazione di memoria $1$, in cui si trova la stringa $"10001010"$, all'uscita della macchina M si ottiene $0010$;
   \item \textbf{istante 2:} si accede alla locazione di memoria $2$, in cui si trova la stringa $"00001101"$, procedendo come in precedenza si ottiene $1101$.
\end{itemize}
Si può procedere in questo modo per tutte le locazioni di memoria scandite dal contatore confermando così il risultato della simulazione.\\
Come detto, si è scelto di mostrare anche le variazioni dello stato e del contatore, in modo da avere la possibilità di osservare in ogni istante il comportamento del sistema, tale scelta è del tutto opzionale.
\section{Implementazione su board del punto precedente}
\subsection{Traccia}
Sintetizzare ed implementare su board il componente sviluppato al punto 
precedente, utilizzando due bottoni per i segnali di read e reset rispettivamente e i 
led per la visualizzazione delle uscite della macchina istante per istante.
\subsection{Implementazione}
Nel caso in esame, per procedere all'implementazione su board, è stato necessario apportare alcune modifiche: viene infatti richiesto che la lettura sia abilitata da un segnale esterno proveniente da un bottone della board, mentre nell'implementazione precedente, l'abilitazione alla lettura era un'uscita dell'unità di controllo, che veniva posta alta o bassa in base allo stato corrente. in ogni caso, i codici di ROM, M, counter e MEM restano invariati rispetto al punto precedente, e vengono quindi semplicemente importati.\\
Si dimostra inoltre necessario l'utilizzo di un divisore di frequenza, componente che si importa dagli esercizi precedenti, per cui non si riporta nuovamente il codice.\\
Si procede quindi a mostrare le modifiche apportate all'implementazione dell'unità di controllo e di $ROM\_M\_M.vhdl$ per la gestione dell'$EN\_RD$ come segnale di ingresso dalla board.
\begin{code}
    \inputminted[frame=lines, framesep=2mm, baselinestretch=1.2, bgcolor=LightGray, fontsize=\footnotesize, linenos]{vhdl}{vhdl_files/ROM_M_MEM_board/control_unit.vhd}
    \caption{Control Unit con EN\_RD come ingresso}
    \label{lbl:ROMC}
\end{code}
\begin{code}
    \inputminted[frame=lines, framesep=2mm, baselinestretch=1.2, bgcolor=LightGray, fontsize=\footnotesize, linenos]{vhdl}{vhdl_files/ROM_M_MEM_board/Rom_M_MEM.vhd}
    \caption{ROM + M + MEM su board}
    \label{lbl:ROMC}
\end{code}
Si nota che le uscite commentate nel codice (stato e count) sono state utilizzate ai fini del debugging, per monitorare che le uscite corrispondessero alla giusta locazione di memoria e che i cambiamenti di stato avvenissero in maniera efficiente.\\
Si è scelto inoltre di mostrare attraverso $led 15$ della board anche le variazioni del clock, per un'ulteriore conferma visiva della correttezza del sistema.\\
Come si deduce dal codice mostrato, è stato scelto di usare 3 bottoni per gestire il funzionamento del sistema, $BTNU$ come ingresso per $START$, $BTNL$ come ingresso per $EN\_RD$ e $BNTC$ come ingresso per $reset$; per consentire ciò i collegamenti sono stati aggiunti nel file $Nexys-A7-50T-Master.xdc$.\\
Si mostrano ora i riultati ottenuti sulla board, avendo dato l'abilitazione alla lettura in 3 istanti diversi.
\begin{figure}[H]
	\centering
	\includegraphics[width=0.7\textwidth]{img/Test1_ROM_M_MEMwebp.png}
	\caption{Istante 0: uscita = $1001$}
	\label{schemROM_M_MEM} 
\end{figure}

\begin{figure}[H]
	\centering
	\includegraphics[width=0.7\textwidth]{img/Test2_ROM_M_MEMwebp.png}
	\caption{Istante 2: uscita = $1101$}
	\label{schemROM_M_MEM} 
\end{figure}


\begin{figure}[H]
	\centering
	\includegraphics[width=0.7\textwidth]{img/Test3_ROM_M_MEMwebp.png}
	\caption{Istante 3: uscita = $1100$}
	\label{schemROM_M_MEM} 
\end{figure}

\chapter{Esercizio 7}
\section{Moltiplicatore di Booth}
 Si vuole progettare, implementare in VHDL e simulare il moltiplicatore di Booth, in grado di effettuare il prodotto tra due stringe di 8 bits ciascuna.
 \subsection{Progettazione}
 Prima di progettare le componenti necessari, è necessario progettare per interezza il moltiplicatore:
\begin{figure}[H]
	\centering
	\includegraphics[width=1\textwidth]{img/Esercizio_7_1/booth_prog}
	\caption{Moltiplicatore di Booth}
	\label{booth_prog} 
\end{figure}

Le componenti necessarie sono quindi:
\begin{itemize}
    \item \textbf{Shift\_register}: uno per il valore $A$ (Accumulator), uno per il valore $Q$ e un terzo, il quale memorizzerà il il valore $M$ (quest'ultimo non avrà necessità di fare shift);
    \item \textbf{Flip-Flop D}: necessario per la memorizzazione del valore $Q_1$ e componente strutturale il contatore;
    \item \textbf{Counter}: contatore che porta il conteggio dei passi effettuati;
    \item \textbf{Parallel Adder}: un moltiplicatore parallelo per effettuare la somma: nell'esempio si utilizza un sommatore Carry-Look-Ahead;
    \item \textbf{Control unit}: unità di controllo per la gestione dell'unità operativa.
\end{itemize}

\subsubsection{Shift\_Register}
Lo Shift\_Register utilizzato differisce da quello visto in precedenza per la presenza di due ingressi aggiuntivi che consentono l'inizializzazione del registro e l'inserimento di un'intera stringa nel registro:

\begin{figure}[H]
	\centering
	\includegraphics[width=1\textwidth]{img/Esercizio_7_1/Shift_Register_booth.png}
	\caption{Shift\_Register}
	\label{sh_booth} 
\end{figure}

\subsubsection{Flip-Flop D}
Il Flip-Flop D è lo stesso progettato nel capitolo (Citare capitolo).
Si deve notare, che alcune delle porte di ingresso del Flip-Flop D rimarranno inutilizzate, poiché in tale progetto non vi è la necessità di effettuare un settaggio iniziale.

\subsubsection{Counter}
La macchina counter è un contatore modulo 3.\\
Analogamente ai contatori presentati nell'esercizio del cronometro, esso si ottiene utilizzando 3 Flip-Flop D in parallelo.\\

\begin{figure}[H]
	\centering
	\includegraphics[width=1\textwidth]{img/Esercizio_7_1/counter_mod_8.png}
	\caption{Counter mod 8}
	\label{cnt_mod_8} 
\end{figure}

\subsubsection{Carry-Look-Ahead}
Per effettuare la somma, si necessita un sommatore parallelo.\\ 
Si vuole utilizzare il Carry-Look-Ahead (è possibile scegliere qualsiasi altro sommatore parallelo) con operandi ad 8 bits. 

\begin{figure}[H]
	\centering
	\includegraphics[width=0.7\textwidth]{img/Esercizio_7_1/carry_look_ahead_beh}
	\caption{Carry Look Ahead}
	\label{CLA} 
\end{figure}

Si noti che all'interno di tale sommatore, viene gestita la scelta di operare un'addizione o una sottrazione.

\subsubsection{Control Unit}
Per la gestione delle operazioni, si utilizza una Control Unit.\\
Essa altro non è che una macchina sequenziale e quindi va progettato l'automa a stati finiti corrispondente:

\begin{figure}[H]
	\centering
	\includegraphics[width=1\textwidth]{img/Esercizio_7_1/booth_cu.png}
	\caption{Control Unit}
	\label{cu_automa} 
\end{figure}



\subsection{Implementazione}
\subsubsection{Shift\_Register}
L'implementazione dello Shift Register è la seguente:
 \begin{code}
    \inputminted[frame=lines, framesep=2mm, baselinestretch=1.2, bgcolor=LightGray, fontsize=\footnotesize, linenos]{vhdl}{vhdl_files/Esercizio_7_1/shift_register.vhdl}
    \caption{shift\_register.vhdl}
    \label{lbl:shr}
\end{code}

\subsubsection{Counter}
L'implementazione dello Counter è la seguente:
 \begin{code}
    \inputminted[frame=lines, framesep=2mm, baselinestretch=1.2, bgcolor=LightGray, fontsize=\footnotesize, linenos]{vhdl}{vhdl_files/Esercizio_7_1/counter_mod_8.vhdl}
    \caption{counter\_mod\_8.vhdl}
    \label{lbl:cnt_mod_8}
\end{code}

L'implementazione del Flip-Flop D è la stessa fatta in precedenza. %Citazione


\subsubsection{Carry-Look-Ahead}
L'implementazione del sommatore, con approccio comportamentale, è la seguente:
 \begin{code}
    \inputminted[frame=lines, framesep=2mm, baselinestretch=1.2, bgcolor=LightGray, fontsize=\footnotesize, linenos]{vhdl}{vhdl_files/Esercizio_7_1/carry_look_ahead.vhdl}
    \caption{carry\_look\_ahead.vhdl}
    \label{lbl:cla}
\end{code}

\subsubsection{Control Unit}
Lo sviluppo della Control Unit è esposto di seguito:
 \begin{code}
    \inputminted[frame=lines, framesep=2mm, baselinestretch=1.2, bgcolor=LightGray, fontsize=\footnotesize, linenos]{vhdl}{vhdl_files/Esercizio_7_1/cu.vhdl}
    \caption{cu.vhdl}
    \label{lbl:cu_booth}
\end{code}

\subsubsection{Booth}
Quindi a questo punto, si può implementare strutturalmente la macchina Booth:
 \begin{code}
    \inputminted[frame=lines, framesep=2mm, baselinestretch=1.2, bgcolor=LightGray, fontsize=\footnotesize, linenos]{vhdl}{vhdl_files/Esercizio_7_1/booth.vhdl}
    \caption{booth.vhdl}
    \label{lbl:booth}
\end{code}

\subsection{Simulazione}
Per effettuare la simulazione della macchina, è stato implementato il seguente test\_bench:
 \begin{code}
    \inputminted[frame=lines, framesep=2mm, baselinestretch=1.2, bgcolor=LightGray, fontsize=\footnotesize, linenos]{vhdl}{vhdl_files/Esercizio_7_1/booth_tb.vhdl}
    \caption{booth\_tb.vhdl}
    \label{lbl:tb_booth}
\end{code}

Il risultato è il seguente:

\begin{figure}[H]
	\centering
	\includegraphics[width=1\textwidth]{img/Esercizio_7_1/booth_sim_1.png}
	\caption{Simulazione 1: $3 \times -1$}
	\label{booth_sim_1} 
\end{figure}

\begin{figure}[H]
	\centering
	\includegraphics[width=1\textwidth]{img/Esercizio_7_1/booth_sim_1.png}
	\caption{Simulazione 1: $-13 \times -16$}
	\label{booth_sim_1} 
\end{figure}

\section{Implementazione su board del punto precedente}
\subsection{Traccia}
 Sintetizzare il moltiplicatore implementato al punto 7.1 su FPGA e testarlo mediante l’utilizzo dei dispositivi di input/output (switch, bottoni, led, display) presenti sulla board di sviluppo in dotazione. La modalità di utilizzo degli stessi è a completa discrezione degli 
studenti.
\subsection{Implementazione}
Per l'implementazione su board si è scelto di utilizzare gli switches da 0 a 7 per il primo operando e gli switches da 8 a 15 per il secondo operando. Per lo start si è utilizzato il bottone BTNU, mentre per il reset si è utilizzato il bottone centrale BTNC. Per realizzare ciò è stato utilizzato il seguente file \textit{Nexys-A7-50T-Master.xdc} come mostrato:
{\footnotesize
\begin{verbatim}
# Clock signal
set_property -dict { PACKAGE_PIN E3    IOSTANDARD LVCMOS33 } [get_ports { clk }]; 
#IO_L12P_T1_MRCC_35 Sch=clk100mhz
create_clock -add -name sys_clk_pin -period 100000 -waveform {0 5}
[get_ports {clk}];

##Switches
set_property -dict { PACKAGE_PIN J15   IOSTANDARD LVCMOS33 } 
[get_ports { X[0] }]; #IO_L24N_T3_RS0_15 Sch=sw[0]
set_property -dict { PACKAGE_PIN L16   IOSTANDARD LVCMOS33 } 
[get_ports { X[1]}]; #IO_L3N_T0_DQS_EMCCLK_14 Sch=sw[1]
set_property -dict { PACKAGE_PIN M13   IOSTANDARD LVCMOS33 } 
[get_ports { X[2] }]; #IO_L6N_T0_D08_VREF_14 Sch=sw[2]
set_property -dict { PACKAGE_PIN R15   IOSTANDARD LVCMOS33 } 
[get_ports { X[3] }]; #IO_L13N_T2_MRCC_14 Sch=sw[3]
set_property -dict { PACKAGE_PIN R17   IOSTANDARD LVCMOS33 } 
[get_ports { X[4] }]; #IO_L12N_T1_MRCC_14 Sch=sw[4]
set_property -dict { PACKAGE_PIN T18   IOSTANDARD LVCMOS33 } 
[get_ports { X[5] }]; #IO_L7N_T1_D10_14 Sch=sw[5]
set_property -dict { PACKAGE_PIN U18   IOSTANDARD LVCMOS33 } 
[get_ports { X[6] }]; #IO_L17N_T2_A13_D29_14 Sch=sw[6]
set_property -dict { PACKAGE_PIN R13   IOSTANDARD LVCMOS33 } 
[get_ports { X[7] }]; #IO_L5N_T0_D07_14 Sch=sw[7]
set_property -dict { PACKAGE_PIN T8    IOSTANDARD LVCMOS18 } 
[get_ports { Y[0] }]; #IO_L24N_T3_34 Sch=sw[8]
set_property -dict { PACKAGE_PIN U8    IOSTANDARD LVCMOS18 } 
[get_ports { Y[1] }]; #IO_25_34 Sch=sw[9]
set_property -dict { PACKAGE_PIN R16   IOSTANDARD LVCMOS33 } 
[get_ports { Y[2] }]; #IO_L15P_T2_DQS_RDWR_B_14 Sch=sw[10]
set_property -dict { PACKAGE_PIN T13   IOSTANDARD LVCMOS33 } 
[get_ports { Y[3] }]; #IO_L23P_T3_A03_D19_14 Sch=sw[11]
set_property -dict { PACKAGE_PIN H6    IOSTANDARD LVCMOS33 } 
[get_ports { Y[4] }]; #IO_L24P_T3_35 Sch=sw[12]
set_property -dict { PACKAGE_PIN U12   IOSTANDARD LVCMOS33 } 
[get_ports { Y[5] }]; #IO_L20P_T3_A08_D24_14 Sch=sw[13]
set_property -dict { PACKAGE_PIN U11   IOSTANDARD LVCMOS33 } 
[get_ports { Y[6] }]; #IO_L19N_T3_A09_D25_VREF_14 Sch=sw[14]
set_property -dict { PACKAGE_PIN V10   IOSTANDARD LVCMOS33 } 
[get_ports { Y[7] }]; #IO_L21P_T3_DQS_14 Sch=sw[15]

## LEDs
set_property -dict { PACKAGE_PIN H17   IOSTANDARD LVCMOS33 } 
[get_ports { res[0] }]; #IO_L18P_T2_A24_15 Sch=led[0]
set_property -dict { PACKAGE_PIN K15   IOSTANDARD LVCMOS33 } 
[get_ports { res[1] }]; #IO_L24P_T3_RS1_15 Sch=led[1]
set_property -dict { PACKAGE_PIN J13   IOSTANDARD LVCMOS33 } 
[get_ports { res[2] }]; #IO_L17N_T2_A25_15 Sch=led[2]
set_property -dict { PACKAGE_PIN N14   IOSTANDARD LVCMOS33 } 
[get_ports { res[3] }]; #IO_L8P_T1_D11_14 Sch=led[3]
set_property -dict { PACKAGE_PIN R18   IOSTANDARD LVCMOS33 } 
[get_ports { res[4] }]; #IO_L7P_T1_D09_14 Sch=led[4]
set_property -dict { PACKAGE_PIN V17   IOSTANDARD LVCMOS33 } 
[get_ports { res[5] }]; #IO_L18N_T2_A11_D27_14 Sch=led[5]
set_property -dict { PACKAGE_PIN U17   IOSTANDARD LVCMOS33 } 
[get_ports { res[6] }]; #IO_L17P_T2_A14_D30_14 Sch=led[6]
set_property -dict { PACKAGE_PIN U16   IOSTANDARD LVCMOS33 } 
[get_ports { res[7] }]; #IO_L18P_T2_A12_D28_14 Sch=led[7]
set_property -dict { PACKAGE_PIN V16   IOSTANDARD LVCMOS33 } 
[get_ports { res[8] }]; #IO_L16N_T2_A15_D31_14 Sch=led[8]
set_property -dict { PACKAGE_PIN T15   IOSTANDARD LVCMOS33 } 
[get_ports { res[9] }]; #IO_L14N_T2_SRCC_14 Sch=led[9]
set_property -dict { PACKAGE_PIN U14   IOSTANDARD LVCMOS33 } 
[get_ports { res[10] }]; #IO_L22P_T3_A05_D21_14 Sch=led[10]
set_property -dict { PACKAGE_PIN T16   IOSTANDARD LVCMOS33 } 
[get_ports { res[11] }]; #IO_L15N_T2_DQS_DOUT_CSO_B_14 Sch=led[11]
set_property -dict { PACKAGE_PIN V15   IOSTANDARD LVCMOS33 } 
[get_ports { res[12] }]; #IO_L16P_T2_CSI_B_14 Sch=led[12]
set_property -dict { PACKAGE_PIN V14   IOSTANDARD LVCMOS33 } 
[get_ports { res[13] }]; #IO_L22N_T3_A04_D20_14 Sch=led[13]
set_property -dict { PACKAGE_PIN V12   IOSTANDARD LVCMOS33 } 
[get_ports { res[14] }]; #IO_L20N_T3_A07_D23_14 Sch=led[14]
set_property -dict { PACKAGE_PIN V11   IOSTANDARD LVCMOS33 } 
[get_ports { res[15] }]; #IO_L21N_T3_DQS_A06_D22_14 Sch=led[15]

##Buttons
#set_property -dict { PACKAGE_PIN C12   IOSTANDARD LVCMOS33 } 
[get_ports { reset }]; #IO_L3P_T0_DQS_AD1P_15 Sch=cpu_resetn
set_property -dict { PACKAGE_PIN N17   IOSTANDARD LVCMOS33 } 
[get_ports { rst }]; #IO_L9P_T1_DQS_14 Sch=btnc
set_property -dict { PACKAGE_PIN M18   IOSTANDARD LVCMOS33 } 
[get_ports { start }]; #IO_L4N_T0_D05_14 Sch=btnu
#set_property -dict { PACKAGE_PIN P17   IOSTANDARD LVCMOS33 } 
[get_ports { load_i }]; #IO_L12P_T1_MRCC_14 Sch=btnl
#set_property -dict { PACKAGE_PIN M17   IOSTANDARD LVCMOS33 } 
[get_ports { load_M }]; #IO_L10N_T1_D15_14 Sch=btnr
#set_property -dict { PACKAGE_PIN P18   IOSTANDARD LVCMOS33 } 
[get_ports { BTND }]; #IO_L9N_T1_DQS_D13_14 Sch=btnd

\end{verbatim}
}
Da cui si osserva come sono state collegate le componenti utilizzate.\\
In seguito si mostrano alcune immagini del funzionamento del moltiplicatore sulla board.
\begin{figure}[H]
	\centering
	\includegraphics[width=0.8\textwidth]{img/onBoard/booth_test1}
	\caption{Test 1}
	\label{test1} 
\end{figure}
Nell'immagine rappresentante il Test 1 si è svolta la seguente moltiplicazione:\\
$11000000 * 00000010 = 1111111110000000$\\
che convertita in decimale risulta\\
$-64 * 2 = -128$\\
avendo utilizzato la notazione in complemento a 2.
\begin{figure}[H]
	\centering
	\includegraphics[width=0.8\textwidth]{img/onBoard/booth_test2}
	\caption{Test 2}
	\label{test1} 
\end{figure}
In Test 2 invece si è svolta la seguente moltiplicazione\\
$00000110 * 00000001 = 0000000000000110$ \\
che convertita in decimale risulta\\
$3 * 1 = 3$\\.

\chapter{Esercizio 8}
\section{Prova di esame del 19 dicembre 2024}
\subsection{Traccia}
 Un sistema è composto da 2 nodi, A e B. A include una ROM (progettata come macchina sequenziale con READ sincrono) di 8 locazioni da 4 bit, mentre B include un sommatore 
parallelo in grado di effettuare la somma di 2 stringhe di 4 bit ciascuna e un registro R di 4 bit. Il sistema opera come segue: all’arrivo di un segnale di start,  A inizia a prelevare gli elementi ROM[i] dalla propria memoria e li invia, uno alla volta, a B mediante handshaking. B somma progressivamente le stringhe ricevute utilizzando il sommatore e alla fine inserisce il risultato nel registro R.
\subsection{Progettazione}
Il progetto di questo sistema si compone di due nodi fondamentali: il nodo A e il nodo B che comunicano tramite handshaking.\\
Il nodo A si compone di una memoria di sola lettura (ROM), un contatore e un'unità di controllo che gestisce l'interazione;\\
il nodo B si compone di un sommatore \textit{Carry Look Ahead}, di un registro e di una unittà di controllo per le rispettive operazioni.\\
Di seguito viene mostrata lo schema a blocchi del sistema, con la suddivisione in due nodi distinti.
\begin{figure}[H]
	\centering
	\includegraphics[width=1\textwidth]{img/preappDicembre/Preapp_NodoA}
	\caption{Schema a blocchi: nodo A}
	\label{wf_preapp} 
\end{figure}
\begin{figure}[H]
	\centering
	\includegraphics[width=1\textwidth]{img/preappDicembre/Preapp_NodoB}
	\caption{Schema a blocchi: nodo B}
	\label{wf_preapp} 
\end{figure}
Nella fase di progettazione, si creano anche gli automi relativi alle unità di controllo dei due nodi, qui mostrati:
\begin{figure}[H]
	\centering
	\includegraphics[width=1\textwidth]{img/preappDicembre/Preapp_automaA}
	\caption{Automa unità di controllo: nodo A}
	\label{wf_preapp} 
\end{figure}
\begin{figure}[H]
	\centering
	\includegraphics[width=1\textwidth]{img/preappDicembre/Preapp_automaB}
	\caption{Automa unità di controllo: nodo B}
	\label{wf_preapp} 
\end{figure}
\subsection{Implementazione}
Per l'implementazione si parte dal nodo A: \\
partendo dall'unità operativa, essa è composta da una ROM e da un contatore. Si mostrano i rispettivi codici.
\begin{code}
    \inputminted[frame=lines, framesep=2mm, baselinestretch=1.2, bgcolor=LightGray, fontsize=\footnotesize, linenos]{vhdl}{vhdl_files/preappDicembre/ROM.vhd}
    \caption{ROM.vhdl}
    \label{lst:mux_2_1}
\end{code}

\begin{code}
    \inputminted[frame=lines, framesep=2mm, baselinestretch=1.2, bgcolor=LightGray, fontsize=\footnotesize, linenos]{vhdl}{vhdl_files/preappDicembre/count_8.vhd}
    \caption{Contatore modulo 8.vhdl}
    \label{lst:mux_2_1}
\end{code}
Tali componenti vengono collegati tra loro nell'\textit{unità operativa}:
\begin{code}
    \inputminted[frame=lines, framesep=2mm, baselinestretch=1.2, bgcolor=LightGray, fontsize=\footnotesize, linenos]{vhdl}{vhdl_files/preappDicembre/unita_operativa.vhd}
    \caption{CUnità operativa di A in vhdl}
    \label{lst:mux_2_1}
\end{code}
Per la gestione delle abilitazioni, si utilizza un'unità di controllo, come si vede dallo schema a blocchi nel paragrafo precedente:
\begin{code}
    \inputminted[frame=lines, framesep=2mm, baselinestretch=1.2, bgcolor=LightGray, fontsize=\footnotesize, linenos]{vhdl}{vhdl_files/preappDicembre/UCA.vhd}
    \caption{Control Unit di A.vhdl}
    \label{lst:mux_2_1}
\end{code}
Il nodo A nel suo complesso sarà implementato in questo modo:
\begin{code}
    \inputminted[frame=lines, framesep=2mm, baselinestretch=1.2, bgcolor=LightGray, fontsize=\footnotesize, linenos]{vhdl}{vhdl_files/preappDicembre/A.vhd}
    \caption{nodo A.vhdl}
    \label{lst:mux_2_1}
\end{code}
Si procede ora con l'implementazione del nodo B, le cui componenti sono un registro R per lo storage del risultato e un Carry Look Ahead:
\begin{code}
    \inputminted[frame=lines, framesep=2mm, baselinestretch=1.2, bgcolor=LightGray, fontsize=\footnotesize, linenos]{vhdl}{vhdl_files/preappDicembre/register.vhd}
    \caption{register.vhdl}
    \label{lst:mux_2_1}
\end{code}
Il sommatore è stato realizzato con un approccio strutturale, a partire da full adder:
\begin{code}
    \inputminted[frame=lines, framesep=2mm, baselinestretch=1.2, bgcolor=LightGray, fontsize=\footnotesize, linenos]{vhdl}{vhdl_files/preappDicembre/full_adder.vhd}
    \caption{full\_adder.vhdl}
    \label{lst:mux_2_1}
\end{code}

\begin{code}
    \inputminted[frame=lines, framesep=2mm, baselinestretch=1.2, bgcolor=LightGray, fontsize=\footnotesize, linenos]{vhdl}{vhdl_files/preappDicembre/CarryLookAhead.vhd}
    \caption{CarryLookAhead.vhdl}
    \label{lst:mux_2_1}
\end{code}
Si mostra ora il codice dell'unità operativa:
\begin{code}
    \inputminted[frame=lines, framesep=2mm, baselinestretch=1.2, bgcolor=LightGray, fontsize=\footnotesize, linenos]{vhdl}{vhdl_files/preappDicembre/unita_operativaB.vhd}
    \caption{unità operativa di B in vhdl}
    \label{lst:mux_2_1}
\end{code}
Come già visto, per la gestione delle abilitazioni e del funzionamento si utilizza un unità di controllo, modellata sulla base degli automi progettati nel paragrafo precedente.
\begin{code}
    \inputminted[frame=lines, framesep=2mm, baselinestretch=1.2, bgcolor=LightGray, fontsize=\footnotesize, linenos]{vhdl}{vhdl_files/preappDicembre/UCB.vhd}
    \caption{Control Unit di A.vhdl}
    \label{lst:mux_2_1}
\end{code}
Il nodo B nel suo complesso viene implementato in questo modo:
\begin{code}
    \inputminted[frame=lines, framesep=2mm, baselinestretch=1.2, bgcolor=LightGray, fontsize=\footnotesize, linenos]{vhdl}{vhdl_files/preappDicembre/B.vhd}
    \caption{nodo B.vhdl}
    \label{lst:mux_2_1}
\end{code}
Il sistema compolessivo composto dai due nodi creato in precedenza si implementa come segue:
\begin{code}
    \inputminted[frame=lines, framesep=2mm, baselinestretch=1.2, bgcolor=LightGray, fontsize=\footnotesize, linenos]{vhdl}{vhdl_files/preappDicembre/AplusB.vhd}
    \caption{AplusB.vhdl}
    \label{lst:mux_2_1}
\end{code}
Si osserva anche lo schematic complessivo generato dall'ambiente Vivado:
\begin{figure}[H]
	\centering
	\includegraphics[width=1\textwidth]{img/preappDicembre/schematic_preapp}
	\caption{Schema a blocchi: nodo B}
	\label{wf_preapp} 
\end{figure}
\subsection{Simulazione}
Per procedere alla simulazione, è necessaria un testbench:
\begin{code}
    \inputminted[frame=lines, framesep=2mm, baselinestretch=1.2, bgcolor=LightGray, fontsize=\footnotesize, linenos]{vhdl}{vhdl_files/preappDicembre/testbench.vhd}
    \caption{testbench.vhdl}
    \label{lst:mux_2_1}
\end{code}
E eseguendo tale simulazione si ottiene la seguente waveform:
\begin{figure}[H]
	\centering
	\includegraphics[width=1\textwidth]{img/preappDicembre/waveformPreappDic}
	\caption{Waveform}
	\label{wf_preapp} 
\end{figure}
Per valutare la correttezza della simulazione si ricordano gli elementi di ROM:
\begin{center}
\begin{tabbing}
0000 \=  \kill % Per definire l'allineamento
ROM[0] = 0000 \\
ROM[1] = 1100 \\
ROM[2] = 0010 \\
ROM[3] = 1010 \\
ROM[4] = 0001 \\
ROM[5] = 1111 \\
ROM[6] = 0101 \\
ROM[7] = 0011 \\
\end{tabbing}
\end{center}
Inizializzando il registro R a 0 e poi sommando progressivamente i valori a due a due si ottiene:
\begin{tabbing}
0000 \=  \kill % Per definire l'allineamento
0000 + 0000 = 0000\\
0000 + 1100 = 1100\\
1100 + 0010 = 1110\\
1110 + 1010 = 1000 \\
1000 + 0001 = 1001\\
1001 + 1111 = 1000 \\
1000 + 0101 = 1101 \\
\textbf{1101 + 0011 = 0010} \\
\end{tabbing}
Quindi alla fine sul registro sarà memorizzata la stringa 0010, che corrisponde alla somma di tutti gli elementi della ROM presente in A.
 
%\chapter{Conclusioni}

\lipsum[23]


\bibliography{bibliography}
\bibliographystyle{plain}
\nocite{*}

\end{document}