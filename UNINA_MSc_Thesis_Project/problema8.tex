\chapter{Capitolo 8}
\section{Prova di esame del 19 dicembre 2024}
\subsection{Traccia}
 Un sistema è composto da 2 nodi, A e B. A include una ROM (progettata come macchina sequenziale con READ sincrono) di 8 locazioni da 4 bit, mentre B include un sommatore 
parallelo in grado di effettuare la somma di 2 stringhe di 4 bit ciascuna e un registro R di 4 bit. Il sistema opera come segue: all’arrivo di un segnale di start,  A inizia a prelevare gli elementi ROM[i] dalla propria memoria e li invia, uno alla volta, a B mediante handshaking. B somma progressivamente le stringhe ricevute utilizzando il sommatore e alla fine inserisce il risultato nel registro R.
